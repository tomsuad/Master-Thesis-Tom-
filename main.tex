%%%%%%%%%%%%%%%%%%%%%%%%%%%%%%%%%%%%%%%%%
% Masters/Doctoral Thesis 
% LaTeX Template
% Version 2.4 (22/11/16)
%
% This template has been downloaded from:
% http://www.LaTeXTemplates.com
%
% Version 2.x major modifications by:
% Vel (vel@latextemplates.com)
%
% This template is based on a template by:
% Steve Gunn (http://users.ecs.soton.ac.uk/srg/softwaretools/document/templates/)
% Sunil Patel (http://www.sunilpatel.co.uk/thesis-template/)
%
% Template license:
% CC BY-NC-SA 3.0 (http://creativecommons.org/licenses/by-nc-sa/3.0/)
%
%%%%%%%%%%%%%%%%%%%%%%%%%%%%%%%%%%%%%%%%%

%----------------------------------------------------------------------------------------
%	PACKAGES AND OTHER DOCUMENT CONFIGURATIONS
%----------------------------------------------------------------------------------------

\documentclass[
11pt, % The default document font size, options: 10pt, 11pt, 12pt
oneside, % Two side (alternating margins) for binding by default, uncomment to switch to one side
english, % ngerman for German
singlespacing, % Single line spacing, alternatives: onehalfspacing or doublespacing
% for thesis review - use doublespacing, for final version use singlespacing
%draft, % Uncomment to enable draft mode (no pictures, no links, overfull hboxes indicated)
%nolistspacing, % If the document is onehalfspacing or doublespacing, uncomment this to set spacing in lists to single
%liststotoc, % Uncomment to add the list of figures/tables/etc to the table of contents
%toctotoc, % Uncomment to add the main table of contents to the table of contents
%parskip, % Uncomment to add space between paragraphs
%nohyperref, % Uncomment to not load the hyperref package
headsepline, % Uncomment to get a line under the header
%chapterinoneline, % Uncomment to place the chapter title next to the number on one line
%consistentlayout, % Uncomment to change the layout of the declaration, abstract and acknowledgements pages to match the default layout
]{MastersDoctoralThesis} % The class file specifying the document structure

\usepackage[utf8]{inputenc} % Required for inputting international characters
\usepackage[T1]{fontenc} % Output font encoding for international characters

\usepackage{amsfonts}
\usepackage{mathtools}
\usepackage{amsmath}
\usepackage{boxedminipage} 
\usepackage{xspace} 
%\usepackage{wrapfig}
\usepackage{mleftright} 
\usepackage{xcolor} 
\usepackage{soul} 
\usepackage{amsthm}
\usepackage{amssymb}
\usepackage{amsmath} 
\usepackage{enumitem}
\usepackage{bbm}
%\usepackage{geometry}


\usepackage{palatino} % Use the Palatino font by default
%\usepackage[backend=bibtex,style=numeric,citestyle=numeric,natbib=true]{biblatex} % Use the bibtex backend with the authoryear citation style (which resembles APA)
\usepackage[backend=bibtex,style=alphabetic,citestyle=alphabetic,natbib=true]{biblatex} % Use the bibtex backend with the authoryear citation style (which resembles APA)

\addbibresource{Reference.bib} % The filename of the bibliography

\usepackage[autostyle=true]{csquotes} % Required to generate language-dependent quotes in the bibliography

%----------------------------------------------------------------------------------------
%	MARGIN SETTINGS
%----------------------------------------------------------------------------------------

%Theorem
\newtheorem{theorem}{Theorem}[section]
\geometry{
	paper=a4paper, % Change to letterpaper for US letter
	inner=2.5cm, % Inner margin
	outer=3.8cm, % Outer margin
	bindingoffset=.5cm, % Binding offset
	top=1.5cm, % Top margin
	bottom=1.5cm, % Bottom margin
	%showframe, % Uncomment to show how the type block is set on the page
}

%Claim
\newtheorem{claim}{Claim}[section]
\geometry{
	paper=a4paper, % Change to letterpaper for US letter
	inner=2.5cm, % Inner margin
	outer=3.8cm, % Outer margin
	bindingoffset=.5cm, % Binding offset
	top=1.5cm, % Top margin
	bottom=1.5cm, % Bottom margin
	%showframe, % Uncomment to show how the type block is set on the page
}

%Lemma
\newtheorem{lemma}{Lemma}[section]
\geometry{
	paper=a4paper, % Change to letterpaper for US letter
	inner=2.5cm, % Inner margin
	outer=3.8cm, % Outer margin
	bindingoffset=.5cm, % Binding offset
	top=1.5cm, % Top margin
	bottom=1.5cm, % Bottom margin
	%showframe, % Uncomment to show how the type block is set on the page
}

%Remark
\newtheorem{remark}{Remark}[section]
\geometry{
	paper=a4paper, % Change to letterpaper for US letter
	inner=2.5cm, % Inner margin
	outer=3.8cm, % Outer margin
	bindingoffset=.5cm, % Binding offset
	top=1.5cm, % Top margin
	bottom=1.5cm, % Bottom margin
	%showframe, % Uncomment to show how the type block is set on the page
}

%Observation
\newtheorem{observation}{Observation}[section]
\geometry{
	paper=a4paper, % Change to letterpaper for US letter
	inner=2.5cm, % Inner margin
	outer=3.8cm, % Outer margin
	bindingoffset=.5cm, % Binding offset
	top=1.5cm, % Top margin
	bottom=1.5cm, % Bottom margin
	%showframe, % Uncomment to show how the type block is set on the page
}

%corollary
\newtheorem{corollary}{Corollary}[section]
\geometry{
	paper=a4paper, % Change to letterpaper for US letter
	inner=2.5cm, % Inner margin
	outer=3.8cm, % Outer margin
	bindingoffset=.5cm, % Binding offset
	top=1.5cm, % Top margin
	bottom=1.5cm, % Bottom margin
	%showframe, % Uncomment to show how the type block is set on the page
}


%\newenvironment{claim}[1]{\par\noindent\underline{Claim:}\space#1}{}

%%% Marking - macro
%\newcommand\typo[1]{{\textcolor{blue}{ Typo: {#1}}}\xspace}
\newcommand{\TODO}[1]{{\textcolor{red}{ TODO: {#1}}}\xspace}
\newcommand{\nanat}[1]{{\textcolor{red}{ [Anat's note: {#1}] }}\xspace}
\newcommand{\ntom}[1]{{\textcolor{red}{ [Tom's note: {#1}] }}\xspace}


%\newtheorem{theorem}{Theorem}


\newcommand{\codedim}{k}
\newcommand{\dualcodedim}{k^\bot}
\newcommand{\codedist}{d}
\newcommand{\secret}{s}
\newcommand{\shareVec}[1]{\mathbf{sh_{#1}}}
\newcommand{\shareSubVec}[2]{sh_{#1}^{#2}}
\newcommand{\generateMatrix}{G}

\renewcommand{\vec}[1]{\mathbf{#1}}

\newcommand{\BadNoZero}[2]{\textnormal{BadNon0}_{#1,#2}}
\newcommand{\GoodNoZero}[2]{\textnormal{GoodNon0}_{#1,#2}}
\newcommand{\Bad}[2]{\textnormal{Bad}_{#1,#2}}
\newcommand{\SBig}[2]{\textnormal{Big}_{#1,#2}}


\newcommand{\Ent}[1]{\mathbf{I}({#1})}
\newcommand{\rowspan}[1]{\mathbf{rowspan}(#1)}

\newcommand{\myvec}[1]{\mathbf{#1}}
\newcommand{\Leak}[2]{\mathrm{Leak}_{#1}(#2)}
\newcommand{\freeind}[4]{\mathrm{free} \left( #1;#2,#3,#4 \right) }
\newcommand{\freei}{\mathrm{free_i}}
\newcommand{\freeione}{\mathrm{free_1}}
\newcommand{\freeitwo}{\mathrm{free_2}}


\newcommand{\myvec}[1]{\mathbf{#1}}
\newcommand{\Leak}[2]{\mathrm{Leak}_{#1}(#2)}
\newcommand{\freeind}[4]{\mathrm{free} \left( #1;#2,#3,#4 \right) }
\newcommand{\freei}{\mathrm{free_i}}
\newcommand{\freeione}{\mathrm{free_1}}
\newcommand{\freeitwo}{\mathrm{free_2}}
%\renewcommand{\SD}[2]{\mathrm{SD}(#1,#2)}
\newcommand{\F}{\mathbb{F}}
\newcommand{\Ent}[1]{\mathbf{I}({#1})}
\newcommand{\rowspan}[1]{\mathbf{rowspan}(#1)}
%\newcommand{\code}[3]{[#1,#2,#3]\text{-code}}
\newcommand{\mdscode}[3]{[#1,#2,#3]\text{-MDS}}
%\newcommand{\BadI}[2]{\text{Bad}_{#1,#2}}
%\newcommand{\Bad}[2]{\text{Bad}_{#1,#2}}
%\newcommand{\Good}[2]{\text{Good}_{#1,#2}}
\newcommand{\codedim}{k}
\renewcommand{\vec}[1]{\mathbf{#1}}


%%% Common commands
\newcommand{\pred}[1]{\ensuremath{\mathsf{#1}}\xspace}
\newcommand{\SD}[2]{\ensuremath{\pred{SD} \left({#1},{#2} \right)}\xspace}
\newcommand{\Enc}{\mathsf{Enc}}
\newcommand{\Dec}{\mathsf{Dec}}
\newcommand{\Ext}{\mathsf{Ext}}
\newcommand{\zo}{\ensuremath{{\{0,1\}}}\xspace}
\newcommand{\defeq}[0]{\ensuremath{{\;\vcentcolon=\;}}\xspace}
\newcommand{\drawn}{\ensuremath{\xleftarrow{\$}}\xspace}
\newcommand{\wt}{\ensuremath{\mathsf{wt}}\xspace}
\newcommand{\hd}{\ensuremath{\mathsf{HD}}\xspace}
\newcommand{\draw}{\leftarrow}
\newcommand{\eps}{\epsilon}



\newcommand{\ie}{i.e.}


\newcommand{\ip}[1]{\langle#1\rangle}

%% Black-board Alphabets 
\newcommand{\A}{\mathbb{A}\xspace}
\newcommand{\B}{\mathbb{B}\xspace}
\newcommand{\C}{\mathbb{C}\xspace}
\newcommand{\D}{\mathbb{D}\xspace}
\newcommand{\E}{\mathbb{E}\xspace}
\newcommand{\F}{\mathbb{F}\xspace}
\newcommand{\G}{\mathbb{G}\xspace}
\newcommand{\H}{\mathbb{H}\xspace}
\newcommand{\I}{\mathbb{I}\xspace}
\newcommand{\J}{\mathbb{J}\xspace}
\newcommand{\K}{\mathbb{K}\xspace}
\newcommand{\L}{\mathbb{L}\xspace}
\newcommand{\M}{\mathbb{M}\xspace}
\newcommand{\N}{\mathbb{N}\xspace}
\newcommand{\O}{\mathbb{O}\xspace}
\newcommand{\P}{\mathbb{P}\xspace}
\newcommand{\Q}{\mathbb{Q}\xspace}
\newcommand{\R}{\mathbb{R}\xspace}
\newcommand{\S}{\mathbb{S}\xspace}
\newcommand{\T}{\mathbb{T}\xspace}
\newcommand{\U}{\mathbb{U}\xspace}
\newcommand{\V}{\mathbb{V}\xspace}
\newcommand{\W}{\mathbb{W}\xspace}
\newcommand{\X}{\mathbb{X}\xspace}
\newcommand{\Y}{\mathbb{Y}\xspace}
\newcommand{\Z}{\mathbb{Z}\xspace}



%% Cal Alphabets
\newcommand{\cA}{\ensuremath{{\mathcal A}}\xspace}
\newcommand{\cB}{\ensuremath{{\mathcal B}}\xspace}
\newcommand{\cC}{\ensuremath{{\mathcal C}}\xspace}
\newcommand{\cD}{\ensuremath{{\mathcal D}}\xspace}
\newcommand{\cE}{\ensuremath{{\mathcal E}}\xspace}
\newcommand{\cF}{\ensuremath{{\mathcal F}}\xspace}
\newcommand{\cG}{\ensuremath{{\mathcal G}}\xspace}
\newcommand{\cH}{\ensuremath{{\mathcal H}}\xspace}
\newcommand{\cI}{\ensuremath{{\mathcal I}}\xspace}
\newcommand{\cJ}{\ensuremath{{\mathcal J}}\xspace}
\newcommand{\cK}{\ensuremath{{\mathcal K}}\xspace}
\newcommand{\cL}{\ensuremath{{\mathcal L}}\xspace}
\newcommand{\cM}{\ensuremath{{\mathcal M}}\xspace}
\newcommand{\cN}{\ensuremath{{\mathcal N}}\xspace}
\newcommand{\cO}{\ensuremath{{\mathcal O}}\xspace}
\newcommand{\cP}{\ensuremath{{\mathcal P}}\xspace}
\newcommand{\cQ}{\ensuremath{{\mathcal Q}}\xspace}
\newcommand{\cR}{\ensuremath{{\mathcal R}}\xspace}
\newcommand{\cS}{\ensuremath{{\mathcal S}}\xspace}
\newcommand{\cT}{\ensuremath{{\mathcal T}}\xspace}
\newcommand{\cU}{\ensuremath{{\mathcal U}}\xspace}
\newcommand{\cV}{\ensuremath{{\mathcal V}}\xspace}
\newcommand{\cW}{\ensuremath{{\mathcal W}}\xspace}
\newcommand{\cX}{\ensuremath{{\mathcal X}}\xspace}
\newcommand{\cY}{\ensuremath{{\mathcal Y}}\xspace}
\newcommand{\cZ}{\ensuremath{{\mathcal Z}}\xspace}


%% Bold Alphabets
\newcommand{\bA}{\ensuremath{{\mathbf A}}\xspace}
\newcommand{\bB}{\ensuremath{{\mathbf B}}\xspace}
\newcommand{\bC}{\ensuremath{{\mathbf C}}\xspace}
\newcommand{\bD}{\ensuremath{{\mathbf D}}\xspace}
\newcommand{\bE}{\ensuremath{{\mathbf E}}\xspace}
\newcommand{\bF}{\ensuremath{{\mathbf F}}\xspace}
\newcommand{\bG}{\ensuremath{{\mathbf G}}\xspace}
\newcommand{\bH}{\ensuremath{{\mathbf H}}\xspace}
\newcommand{\bI}{\ensuremath{{\mathbf I}}\xspace}
\newcommand{\bJ}{\ensuremath{{\mathbf J}}\xspace}
\newcommand{\bK}{\ensuremath{{\mathbf K}}\xspace}
\newcommand{\bL}{\ensuremath{{\mathbf L}}\xspace}
\newcommand{\bM}{\ensuremath{{\mathbf M}}\xspace}
\newcommand{\bN}{\ensuremath{{\mathbf N}}\xspace}
\newcommand{\bO}{\ensuremath{{\mathbf O}}\xspace}
\newcommand{\bP}{\ensuremath{{\mathbf P}}\xspace}
\newcommand{\bQ}{\ensuremath{{\mathbf Q}}\xspace}
\newcommand{\bR}{\ensuremath{{\mathbf R}}\xspace}
\newcommand{\bS}{\ensuremath{{\mathbf S}}\xspace}
\newcommand{\bT}{\ensuremath{{\mathbf T}}\xspace}
\newcommand{\bU}{\ensuremath{{\mathbf U}}\xspace}
\newcommand{\bV}{\ensuremath{{\mathbf V}}\xspace}
\newcommand{\bW}{\ensuremath{{\mathbf W}}\xspace}
\newcommand{\bX}{\ensuremath{{\mathbf X}}\xspace}
\newcommand{\bY}{\ensuremath{{\mathbf Y}}\xspace}
\newcommand{\bZ}{\ensuremath{{\mathbf Z}}\xspace}


%% Black-board Bold Alphabets
\newcommand{\bbA}{\ensuremath{{\mathbb A}}\xspace}
\newcommand{\bbB}{\ensuremath{{\mathbb B}}\xspace}
\newcommand{\bbC}{\ensuremath{{\mathbb C}}\xspace}
\newcommand{\bbD}{\ensuremath{{\mathbb D}}\xspace}
\newcommand{\bbE}{\ensuremath{{\mathbb E}}\xspace}
\newcommand{\bbF}{\ensuremath{{\mathbb F}}\xspace}

\newcommand{\bbG}{\ensuremath{{\mathbb G}}\xspace}
\newcommand{\bbH}{\ensuremath{{\mathbb H}}\xspace}
\newcommand{\bbI}{\ensuremath{{\mathbb I}}\xspace}
\newcommand{\bbJ}{\ensuremath{{\mathbb J}}\xspace}
\newcommand{\bbK}{\ensuremath{{\mathbb K}}\xspace}
\newcommand{\bbL}{\ensuremath{{\mathbb L}}\xspace}
\newcommand{\bbM}{\ensuremath{{\mathbb M}}\xspace}
\newcommand{\bbN}{\ensuremath{{\mathbb N}}\xspace}
\newcommand{\bbO}{\ensuremath{{\mathbb O}}\xspace}
\newcommand{\bbP}{\ensuremath{{\mathbb P}}\xspace}
\newcommand{\bbQ}{\ensuremath{{\mathbb Q}}\xspace}
\newcommand{\bbR}{\ensuremath{{\mathbb R}}\xspace}
\newcommand{\bbS}{\ensuremath{{\mathbb S}}\xspace}
\newcommand{\bbT}{\ensuremath{{\mathbb T}}\xspace}
\newcommand{\bbU}{\ensuremath{{\mathbb U}}\xspace}
\newcommand{\bbV}{\ensuremath{{\mathbb V}}\xspace}
\newcommand{\bbW}{\ensuremath{{\mathbb W}}\xspace}
\newcommand{\bbX}{\ensuremath{{\mathbb X}}\xspace}
\newcommand{\bbY}{\ensuremath{{\mathbb Y}}\xspace}
\newcommand{\bbZ}{\ensuremath{{\mathbb Z}}\xspace}



%----------------------------------------------------------------------------------------
%	THESIS INFORMATION
%----------------------------------------------------------------------------------------

\thesistitle{Leakage-resilient Secret Sharing} % Your thesis title, this is used in the title and abstract, print it elsewhere with \ttitle
\supervisor{Dr. Anat Paskin-Cherniavsky} % Your supervisor's name, this is used in the title page, print it elsewhere with \supname
\examiner{} % Your examiner's name, this is not currently used anywhere in the template, print it elsewhere with \examname
\degree{Master} % Your degree name, this is used in the title page and abstract, print it elsewhere with \degreename
\author{Tom Suad} % Your name, this is used in the title page and abstract, print it elsewhere with \authorname
\addresses{} % Your address, this is not currently used anywhere in the template, print it elsewhere with \addressname

\subject{Computer Science} % Your subject area, this is not currently used anywhere in the template, print it elsewhere with \subjectname
\keywords{Keyword1, Keyword2, Keyword3} % Keywords for your thesis, this is not currently used anywhere in the template, print it elsewhere with \keywordnames
\university{\href{http://www.university.com}{Ariel University}} % Your university's name and URL, this is used in the title page and abstract, print it elsewhere with \univname
\department{\href{http://department.university.com}{Department of Computer Science}} % Your department's name and URL, this is used in the title page and abstract, print it elsewhere with \deptname
\group{\href{http://researchgroup.university.com}{ }} % Your research group's name and URL, this is used in the title page, print it elsewhere with \groupname
\faculty{\href{http://faculty.university.com}{Faculty of Natural Sciences}} % Your faculty's name and URL, this is used in the title page and abstract, print it elsewhere with \facname

\AtBeginDocument{
\hypersetup{pdftitle=\ttitle} % Set the PDF's title to your title
\hypersetup{pdfauthor=\authorname} % Set the PDF's author to your name
\hypersetup{pdfkeywords=\keywordnames} % Set the PDF's keywords to your keywords
}

\begin{document}

\frontmatter % Use roman page numbering style (i, ii, iii, iv...) for the pre-content pages

\pagestyle{plain} % Default to the plain heading style until the thesis style is called for the body content

%----------------------------------------------------------------------------------------
%	TITLE PAGE
%----------------------------------------------------------------------------------------

\begin{titlepage}
\begin{center}

\vspace*{.06\textheight}
{\scshape\LARGE \univname\par}\vspace{1.5cm} % University name
\textsc{\Large Master Thesis}\\[0.5cm] % Thesis type

\HRule \\[0.4cm] % Horizontal line
{\huge \bfseries \ttitle\par}\vspace{0.4cm} % Thesis title
\HRule \\[1.5cm] % Horizontal line
 
\begin{minipage}[t]{0.4\textwidth}
\begin{flushleft} \large
\emph{Author:}\\
\authorname
%\href{http://www.johnsmith.com}{\authorname} % Author name - remove the \href bracket to remove the link
\end{flushleft}
\end{minipage}
\begin{minipage}[t]{0.4\textwidth}
\begin{flushright} \large
\emph{Supervisor:} \\
\supname %\href{http://www.jamessmith.com}{\supname} % Supervisor name - remove the \href bracket to remove the link  
\end{flushright}
\end{minipage}\\[3cm]
 
\vfill
%\large \textit{A thesis submitted in partial fulfillment of the requirements\\ for the degree of \degreename}\\[0.3cm] % University requirement text
%\textit{in the}\\[0.4cm]
\groupname\\\deptname\\[2cm] % Research group name and department name
 
\vfill

{\large \today}\\[4cm] % Date
\includegraphics[width=50mm]{Figures/Ariel1.png} % University/department logo - uncomment to place it
 
\vfill
\end{center}
\end{titlepage}

%----------------------------------------------------------------------------------------
%	DECLARATION PAGE
%----------------------------------------------------------------------------------------

\begin{declaration}
\addchaptertocentry{\authorshipname} % Add the declaration to the table of contents
\noindent I, \authorname, hereby declare that this thesis entitled, \enquote{\ttitle} and the work presented in it are my own. I confirm that:

\begin{itemize} 
\item This work was done wholly or mainly while in candidature for a research degree at this University.
\item Where any part of this thesis has previously been submitted for a degree or any other qualification at this University or any other institution, this has been clearly stated.
\item Where I have consulted the published work of others, this is always clearly attributed.
\item Where I have quoted from the work of others, the source is always given. With the exception of such quotations, this thesis is entirely my own work.
\item I have acknowledged all main sources of help.
\item Where the thesis is based on work done by myself jointly with others, I have made clear exactly what was done by others and what I have contributed myself.\\
\end{itemize}
 
\noindent Signed:\\
\rule[0.5em]{25em}{0.5pt} % This prints a line for the signature
 
\noindent Date:\\
\rule[0.5em]{25em}{0.5pt} % This prints a line to write the date
\end{declaration}

%\cleardoublepage

%----------------------------------------------------------------------------------------
%	QUOTATION PAGE
%----------------------------------------------------------------------------------------

%\vspace*{0.2\textheight}

%\noindent\enquote{\itshape Thanks to my solid academic training, today I can write hundreds of words on virtually any topic without possessing a shred of information, which is how I got a good job in journalism.}\bigbreak

%\hfill Dave Barry

%----------------------------------------------------------------------------------------
%	DEDICATION
%----------------------------------------------------------------------------------------

\dedicatory{Dedicated to my loving parents Ofer and Sigalit Suad, for their continuous and unparalleled love, help and support. And my sister Ziv, for always being there for me as a friend.} 

%----------------------------------------------------------------------------------------
%	ACKNOWLEDGEMENTS
%----------------------------------------------------------------------------------------

\begin{acknowledgements}
\addchaptertocentry{\acknowledgementname} % Add the acknowledgements to the table of contents

I would like to thank Ariel University and Ariel Cyber Innovation Center for the fellowship which support me while conducting my research.

I also would like to thank my supervisor, who assisted and guided me throughout this research and the degree work: Dr. Anat Paskin-Cherniavsky. 

Finally, I would also like to thank my family for their amazing support throughout the years and the degrees, without you these accomplishments would never have come to fruition.




\end{acknowledgements}




%----------------------------------------------------------------------------------------
%	LIST OF CONTENTS/FIGURES/TABLES PAGES
%----------------------------------------------------------------------------------------

\tableofcontents % Prints the main table of contents
%\listoffigures % Prints the list of figures
%\listoftables % Prints the list of tables

%----------------------------------------------------------------------------------------
%	ABSTRACT PAGE
%----------------------------------------------------------------------------------------

\begin{abstract}
\addchaptertocentry{\abstractname} % Add the abstract to the table of contents
% \include{Chapters/abstract}
The security of cryptographic primitives typically relies on the storage of private secrets by each participant in a perfect manner. 
However, increasingly, side-channel attacks are demonstrating the pitfalls of assuming these cryptographic entities as opaque monolithic objects over the entire duration the primitive remains alive. 
Motivated by such concerns, there is a significant interest in revisiting well-established cryptographic primitives and their implementations to identify whether their security continues to hold in the presence of such side-channel attacks. 

%Fundamental primitives, like secret-sharing schemes when performed over carelessly chosen finite fields lead to devastating security breaches. 
%For example, linear secret sharing schemes over characteristic 2 fields are susceptible to an adversary who performs only one-bit leakage from the shares of all the parties. 
Although there are compilers to convert any secret sharing scheme into one that is robust to local leakage on each of their shares, it is not feasible to replace every instance of traditional secret sharing schemes in use with a leakage-resilient counterpart. 
Beyond efficiency considerations, there may be an appropriate structure in specific secret-sharing schemes that are fundamental to their usage in a particular context.
%Let alone efficiency considerations, there is a lot of structure in certain secret-sharing schemes that are fundamental to their usage in certain context. 
%For example, the multiplication friendly nature of Reed-Solomon or Algebraic Geometry codes is inherent to their use in secure computation protocols. 
For example, the use of a linear secret sharing scheme helps perform secure aggregation of statistics in parallel (for example, the sum of the private inputs of the participants) even in the presence of malicious parties. 
The reconstruction threshold of these secret sharing schemes determines the threshold of corruption permissible in the secure computation protocol; lower reconstruction threshold implies a higher efficiency. 

This paper makes a two-fold contribution. First, we continue the study the local leakage resilience of Reed-Solomon codes as initiated by Benhamouda, Degwekar, Ishai, and Rabin (2018). 
We improve their lower bound on the reconstruction threshold for Reed Solomon codes from $0.907n$ to $0.867n$ for one-bit leakage from each secret share,  where $n$ represents the number of parties receiving the secret shares. 

Next, we explore whether, in the presence of local leakage, there is something inherent to maximum-distance separable (MDS) codes (Reed Solomon code is a particular example from this class of codes) that innately demands high reconstruction thresholds. 
Towards this investigation, we study random MDS codes and their necessary reconstruction threshold to remain resilient to a constant local leakage from each share. 
Given any $\delta\in(0,1/2),$ we prove that most random MDS codes over suitably large fields with reconstruction threshold $(1/2 + \delta)n$ are resilient to constant local leakage.

In terms of techniques, both results rely on a Fourier-analytic approach to this problem. 
In particular, the second result relies on new and subtle analysis techniques for random MDS codes, which we believe shall be of independent interest. 

Finally, we also contribute to the impossibility of designing secret-sharing schemes based on MDS codes over prime-order fields, where the dimension of the code is very small. 
If one insists on exponentially small indistinguishability among the shares generated by two different secrets, then the dimension of the code needs to be $\Omega(n/\log n)$, even when the adversary obtains only $m=1$ bit leakage from each of the shares and the field size is arbitrarily large. 


\end{abstract}


%----------------------------------------------------------------------------------------
%	ABBREVIATIONS
%----------------------------------------------------------------------------------------

\begin{abbreviations}{ll} % Include a list of abbreviations (a table of two columns)

\textbf{MPC} & \textbf{M}ulti-\textbf{P}arty \textbf{C}omputation\\
\textbf{SSS} & \textbf{S}ecret \textbf{S}haring \textbf{S}cheme\\
\textbf{LSSS} & \textbf{L}eakage-resilient \textbf{S}ecret \textbf{S}haring \textbf{S}cheme\\
\textbf{MDS} & \textbf{M}aximum \textbf{D}istance \textbf{S}eparable\\
\textbf{ECC} & \textbf{E}rror-\textbf{C}orrecting \textbf{C}ode\\

\end{abbreviations}

%----------------------------------------------------------------------------------------
%	PHYSICAL CONSTANTS/OTHER DEFINITIONS
%----------------------------------------------------------------------------------------


%----------------------------------------------------------------------------------------
%	SYMBOLS
%----------------------------------------------------------------------------------------

\begin{symbols}{lll} % Include a list of Symbols (a three column table)

$\codedim$ & Dimension of ECC \\
$\codedist$ & Distance of ECC \\
$C$ & Random MDS code \\
$C^\bot$ & Dual code of $C$ \\ \\
$\secret$ & Secret \\
$\shareVec$ & Sequence of shares \\
$n$ & Number of parties \\
$t$ & Threshold of the SSS \\ \\
$\F$ & Field \\
$p$ & Prime number \\
$\F_p$ & Field of order $p$ \\ \\
$\bbC$ & Complex numbers \\ 

\end{symbols}




%----------------------------------------------------------------------------------------
%	THESIS CONTENT - CHAPTERS
%----------------------------------------------------------------------------------------

\mainmatter % Begin numeric (1,2,3...) page numbering

\pagestyle{thesis} % Return the page headers back to the "thesis" style

% Include the chapters of the thesis as separate files from the Chapters folder
% Uncomment the lines as you write the chapters
% Introduction
\include{Chapters/introduction}

% Preliminaries
%\chapter {Preliminary Results}
%\section {Theoretical Results}
%\section {Simulation Results}

\chapter{Preliminaries}
%\section{Preliminaries}
\label{sec:prelim}


We denote by $\Ent{p}=-p\cdot \ln(p)-(1-p)\cdot \ln(1-p)$ the Shannon entropy of $0<p<1$.
For $j\leq k$, where $(n-j)/n, j/n=\Theta(1)$ it follows from Stirling's approximation that
\begin{align}
&\binom{n}{j}=\left(1+o(1)\right)2^{\Ent{j/n}n}\label{eq:bin}
\end{align}
By $\log(x)$ we refer to the base 2 logarithm unless stated otherwise. 

Given two matrices $M_1\in\F^{a_1\times b}, M_2\in\F^{a_2\times b}$, we denote by $(M_1;M_2)$
the matrix resulting from concatenating $M_2$ under $M_1$, \ie, the matrix $\smalleft[\begin{smallmatrix}M_1\\M_2\end{smallmatrix}\smarightright]$.
We denote the number of rows (columns) in $M_1$ by $rows(M)$ ($cols(M)$).By default, vectors are row vectors (which are sometimes viewed as $1\times a$ matrices). A submatrix corresponding to index sets of rows and columns $X,Y$ respectively is denoted by $M[X,Y]$. In this context, `*' stands for the set of all rows or columns respectively, and abbreviate singleton sets via the element contained in it. We sometimes use $M[X]$ as an abbreviation for $M[X,*]$. Similarly, for a vector $v$, $v[I]$ denotes the vector resulting from projecting $v$ to a subset $I$ of its coordinates. For a set of vectors $\mathcal{G}\subseteq A^r$, and a set of indices $I\subseteq [r]$, we also denote $\mathcal{G}[I]=\{a[I] \;|\; a\in \mathcal{G}\}$.  
%Sets of size 1 are abbreviated as the element itself.

For a set $A$, when there is no risk of confusion, we sometimes abuse notation, and view $A$ as the uniform distribution over $A$.



\section{Error correcting codes.} 
An $[n,\codedim,\F_p]$ linear error-correcting code (ECC) $C$ is a subspace of $\F^n_p$ of dimension $\codedim$. 
A code is said to have distance $\codedist$ if every pair of distinct codewords have Hamming distance at least $\codedist$. A generating matrix $\generateMatrix = (g_1;...;g_n) \in\F^{n\times k}_p$ of $C$ is a matrix whose columns constitute a basis of $C$. The dual code $C^\bot$ of a linear code $C$ is $\{x\in \F^n_p \;|\; \forall c \in C,\ip{x,c} = 0\}$, and we denote the generating matrix of $C^\bot$ by $H=(h_1;\ldots;h_n)\in \F^{n\times (n-k)}_p$.

We say a code is Maximum Distance Separable (MDS) if $\codedist=n-\codedim+1$. %We use the following facts about MDS codes. 
 %$C^\bot$ is also a linear $[n,\codedim'=n-\codedim,\F_p]$ code.
We will need a few well known equivalent formulations of MDS codes.
\begin{claim}
Let $C$ be a linear $[n,\codedim,\F_p]$ code. Then the following statements are equivalent:
\begin{itemize}
    \item $C$ is a linear $\mdscode{n}{\codedim}{\F_p}$ code.
    \item $C^\bot$ is a linear $\mdscode{n}{\codedim'=n-\codedim}{\F_p}$ code.
    \item Every set of $\codedim$ rows of $\generateMatrix$ are linearly independent.
\end{itemize}
\end{claim}

\noindent 


\section{Leakage resilient secret sharing}
% Macro usage example: \shareSubVec{A_7}
We consider the standard notion of perfect threshold secret sharing schemes. Namely, an $(n,t)$-secret sharing scheme over a finite field $\F_p$ is a pair of algorithms $(Sh,Rec)$, where $Sh$ is a randomized mapping taking a secret $\secret \in \F_p$ to a sequence of shares $\shareVec{}=(\shareSubVec{}{1},\ldots,\shareSubVec{}{n})$. It is correct in the sense that for each $A\subseteq [n]$ of size $|A|\geq t$ and $\shareVec{}\leftarrow Sh(\secret)$, it hold that $Rec(A,\shareSubVec{}{A}=(\shareSubVec{}{i})_{i\in A})=\secret$ with probability 1 (over the random choices of $Sh$). It is private in the sense that for $\shareVec{0} \leftarrow Sh(\secret_0),\shareVec{1}\leftarrow Sh(\secret_1)$, and every $A\subseteq [n]$ of size $|A|<t$, $\SD{\shareSubVec{0}{A}}{\shareSubVec{1}{A}}=0$. We say a scheme is linear, if in addition, for every share $\shareSubVec{}{i}$ comes from a domain $S_i=\F^{l_i}_p$ for some $l_i\in \mathbb{N}^+$, and for each $a,b,\secret_0,\secret_1\in \F_p$, and valid sharings
$\shareVec{0},\shareVec{1}$ respectively, it holds that $a\shareVec{0}+b\shareVec{1}\in support(Sh(a \secret_0+b \secret_1))$.


In this work, we study the leakage resilience of secret sharing schemes arising from MDS codes in a natural way as in Massey's construction. That is, given a linear $\mdscode{n+1}{\codedim=t}{\F_p}$ code $C^+$ with generating matrix $\generateMatrix^+ = (g^+_1;...;g^+_{n+1})$, the corresponding (linear) $(n,t)$-secret sharing scheme over $\F_p$ is defined as,
$Sh(\secret)$ samples a random vector $\beta=(\beta_1,\ldots,\beta_t)\in\F^t_p$ conditioned on $(\generateMatrix^+ \cdot \beta^T)[n+1]=\secret$, and sets $\shareSubVec{}{i}=\ip{\beta,g^+_i}$. For $|I| \geq t$ and the share vector $\shareSubVec{}{I}$ corresponding to parties $I\subseteq [n]$, $Rec(I,\shareSubVec{}{I})$ takes a subset $I'\subseteq I$ of size $t$, computes $\beta = (G^+[I'])^{-1}{(\shareSubVec{}{I'})}^T$, and outputs $g^+_{n+1}\cdot \beta^T$. We observe that the set of sharings of $\secret=0$ is also a linear MDS code $C$, with parameters $[n,\codedim-1,\F_p]$.
%Indeed $H[I]$ is invertible by MDS properties of the code. Privacy against any subset $I$ of size $t-1$ also easily follows from the fact that $H[I\cup\{i\}]$ is invertible by simple linear algebra.

Here and elsewhere, when we refer to~\cite{EPRINT:BDIR19}, we refer to their eprint version~\cite{EPRINT:BDIR19}.
We follow the definition of local leakage resilient secret sharing schemes as in~\cite{EPRINT:BDIR19}, Definition 4.1.  For completeness, we briefly recall this notion, restricted to MDS code based schemes as above. For a secret sharing scheme $(Sh_C,Rec_C)$ we denote ${\mathcal L}_{m,n,p}$ to be the set of all functions $\F^n_p\to \left(\zo^m\right)^n$ representing $m$-bit local leakage on each share of the $n$ parties.
We say the scheme $(Sh_C,Rec_C)$
is a $(m,\epsilon)$-Leakage resilient secret sharing scheme if for all leakage function vectors $\vec{L}=(L_1,\ldots,L_n) \in {\mathcal L}_{m,n,p}$,
and all pairs of secrets $\secret_0,\secret_1\in\F_p$, we have $\SD{\vec{L}(Sh_C(\secret_0))}{\vec{L}(Sh_C(\secret_1))}\leq \epsilon$.

For a party $P_i$, and $\ell_i\in \{0,1\}^m$, let $A_{i,\ell_i}=\{ x \in \F_p | L_i(x) = \ell_i \}$. We denote by $\vec{1}_{A_i,\ell_i}(x):\F_p\rightarrow \mathbb{C}$ the boolean function mapping $x$ to $1$ if $x\in A_{i,\ell_i}$ and to $0$ otherwise.
When $i$ is clear from the context we sometimes abuse notation and let $\vec{1}_{\ell_i}$ denote $\vec{1}_{A_{i,\ell_i}}$.

For simplicity, our definition above corresponds to the setting with $\Theta=0$ as considered in~\cite{EPRINT:BDIR19}, where the adversary does not see any parties' shares, but only the output of the leakage function, so we exclude $\Theta$ from the notation.
Our results can be translated into resilience for other values with a certain loss in parameters of $\Theta$ as explained in~\cite{EPRINT:BDIR19}.


It is observed in~\cite{EPRINT:BDIR19} that given a leakage functions vector $\vec{L} = (L_1,\ldots,L_n)$, the leakage advantage of the scheme is upper bounded by the statistical distance of $\vec{L}$ applied to a certain pair of distributions.

\begin{observation}\label{obs:bound}
Let $(Sh_C,Rec_C)$ denote an $n$-player Massey secret sharing scheme over $\F_p$. Fix an integer $m\leq \log(p)$. Then for $m$-bit local leakage functions vector $\vec{L}\in {\mathcal{L}}_{m,n,p}$, we have: 
$\max_{\vec{L},\secret_0,\secret_1} \SD{\vec{L}(Sh(\secret_0))}{\vec{L}(Sh(\secret_1))} \leq 2\max_{\vec{L}} \SD{\vec{L}(C)}{\vec{L}(\F^n_p)}$. 
\end{observation}
This the case as for every $\vec{L},\secret_0,\secret_1$,
\begin{align*}
    \SD{\vec{L}(Sh(\secret_0))}{\vec{L}(Sh(\secret_1))} &\leq
    \SD{\vec{L}(C+v_0)}{\vec{L}(\F^n_p)} + \SD{\vec{L}(C+v_1)}{\vec{L}(\F^n_p)} \\
    &\leq 2 \max_{\vec{L}}\SD{\vec{L}(C)}{\vec{L}(\F^n_p)}
\end{align*} 

Here $v_0,v_1$ are some sharings of $\secret_0,\secret_1$ respectively, both are multiplies of the same vector $v$. The first inequality is due to the triangle inequality for the Euclidean space with the $\ell_1$-norm. 
Indeed, by linearity of the scheme, $Sh(\secret_0)$ is uniformly distributed over $C+v_0$ for $v_0=\secret_0v$, for every $\secret_0\in \F_p$.
The last inequality  follows from the claim that $\SD{\vec{L}(C+v_0)}{\vec{L}(\F^n_p)} = \SD{\vec{L}'(C)}{\vec{L}'(\F^n_p)}$ for $\vec{L}'=(L'_1,\ldots,L'_n)$, where $L'_j(x)=L_j(x+s_0 v[j])$ for each $j\in [n]$.  In particular, $\SD{\vec{L}'(C)}{\vec{L}'(\F^n_p)} \leq \max_\vec{L} \SD{\vec{L}(C)}{\vec{L}(\F^n_p)}$. Finally, we get that $\SD{\vec{L}(C+v_0)}{\vec{L}(\F^n_p)} \leq \max_\vec{L} \SD{\vec{L}(C)}{\vec{L}(\F^n_p)}$.
By similar reasoning it holds that $\SD{\vec{L}(C+v_1)}{\vec{L}(\F^n_p)} \leq \max_\vec{L} \SD{\vec{L}(C)}{\vec{L}(\F^n_p)}$, and the claim follows.
\footnote{The last step generalizes the reasoning of Claim 4.8.1 in~\cite{EPRINT:BDIR19}. for additive secret sharing schemes}

%The above observation allows us to reduce bounding local leakage of $(Sh_C,Rec_C)$ to bounding a certain function of the leakage from the uniform distribution on $C$ (although it is not necessarily tight), as this is the type of bound we use throughout  the paper. Thus, for notational convenience, we will sometimes identify between $(Sh_C,Rec_C)$ and $C$.
%\nanat{Think whether we want to talk about a tighter version}
%%%%

We will need the following claim from ~\cite{EPRINT:BDIR19} connecting $\SD{\vec{L}(C)}{\vec{L}(\F^n_p)}$ to evaluations of $\vec{L}$ on elements of $C^\bot \setminus\{0\}$. See Lemma 4.18 ~\cite{EPRINT:BDIR19} for proof details.

\begin{claim} \label{clm:dual}
Let $C$ be a linear $[n,\codedim,\F_p]$ code. Then,
\begin{align}
%%% 2.1 
%\label{eq:3.2}
\SD{\vec{L}(C)}{\vec{L}(\F^n_p)}=\sum_{\vec{\ell}} \left|\sum_{\alpha\in C^{\bot}\setminus{\{0\}}}\prod^n_{i=1}\widehat{\mathbf{1}}_{\ell_i}(\alpha_i) \right|=\sum_{\vec{\ell}}\left|\sum_{\beta\in \F^{n-t+1}_p\setminus{\{0\}}}\prod^n_{i=1}\widehat{\mathbf{1}}_{\ell_i}(\ip{\beta,h_i})\right| \nonumber
\end{align}
\end{claim}


\begin{remark}
The definition of a Massey scheme is straightforwardly generalized to any linear code $C$ with parameters $[n,t]$, which is not
necessarily an MDS code. The resulting secret sharing scheme is not necessarily threshold, and the qualified sets correspond to row subsets $\generateMatrix_I$ for $I\subseteq [n]$ spanning row $\generateMatrix_{n+1}\neq \vec{0}$ (for linear $\mdscode{n+1}{\codedim}{\F_p}$ codes, these are indeed all sets of size $\geq t$, as in such codes sets of size $t$ span every row of the generating matrix), which may be a useful extension. In fact, this type of secret sharing schemes corresponds exactly to ideal linear secret sharing schemes (with $\generateMatrix_{[n]}$ corresponding to matrix and $\generateMatrix_{n+1}$ to the target vector). In fact, both our positive results - those applying to all MDS-induced Massey schemes with given parameters $n,t$ (Theorem~\ref{shamir}) and those applying to a large fraction of such schemes can be extended to work for non-MDS schemes (Theorem~\ref{thm:mostmds}) with a certain loss of parameters, as longs as we have a good lower bound on qualified set size (even if not all are equal). Theorem~\ref{thm-lb} works for non-threshold Massey schemes as is, with the same parameters. See the full version for details.
\end{remark}
%% The last equality follows by observing that replacing a summation over $C$ by summation over $C+v$ for a fixed vector $v$ in the proof of Lemma 4.16 in~\cite{EPRINT:BDIR19} (Poisson Summation Formula), exactly the same expression for $\SD{\vec{L}(C)}{\vec{L}(\F^n_p)}$ is obtained. Thus, in all our results, we focus on bounding the leakage advantage by bounding the latter expression $\SD{\vec{L}(C)}{\vec{L}(\F^n_p)}$.



\section{Fourier Analysis}
%\label{sec:fourier-prelim}
For our purposes, we only recall Fourier analysis for $G$ which is the additive group of a finite field.
Let $\bbF=\{0,\dotsc,p-1\}$ be a field of order $p$, where $p$ is prime.
Let $f \colon \bbF \to \bbC $ be an arbitrary complex-valued function.
For $z\in \bbC$, we let $\overline{z}$ denote the complex conjugate of $z$.
We define the inner-product of two functions $f,g\colon\bbF\to\bbC$ as follows
\begin{equation}\label{eq:norm}
\ip{f,g} \defeq \frac1p \sum_{x\in\bbF} f(x)\overline{g(x)}.
\end{equation}
A character of $\F$ is a homomorphism from the additive group $\F$ to the multiplicative group $\bbC^*$.
The set of characters of $\F$, to which we refer to as $\widehat{\F}$ itself forms a group under coordinate-wise multiplications. In fact $\widehat{\F}=\left\{\chi_0,\chi_1,\dotsc,\chi_{p-1}\right\}$ precisely satisfies
$\chi_i(x) = \exp\left(2\pi \imath \cdot ix/p\right)$. The $\chi_i$'s form an orthonormal basis of
$\bbC^p$. That is, we have:
\begin{align}
\ip{\chi_i,\chi_j} = \begin{cases}0,&\text{if }i\neq j\\1,&\text{if }i=j\end{cases}.
\end{align}
  
%In fact, for $i,x\in\bbF$, the definition of $\chi_i(x) = \exp\left(2\pi \imath \cdot ix/p\right)$ suffices. 
For $i\in\bbF$, we define the Fourier coefficient $\widehat f(i)\defeq \ip{f,\chi_i}$. 
Furthermore, the mapping $f\mapsto \widehat f$ is a full-rank linear mapping. 
Parseval's identity states that
\begin{align}
\ip{f,f} = \sum_{i\in\bbF} \widehat f(i)^2.  
\end{align}
  
As~\cite{EPRINT:BDIR19}, we follow the “standard” notation in additive combinatorics. In this notation, when working over $\F$, the Haar measure as in Equation~\ref{eq:norm}, and the counting measure assigning $1$ to each $i\in\widehat{\F}$ is used when working over $\widehat{\F}$. So, norms will be taken with respect to the underlying measure. Using this convention, we can compactly rephrase Parseval's identity as $\ip{f,f} = \ip{\widehat{f},\widehat{f}}.$

%In general, when considering functions $f,g\colon \bbF^n \to \CC$, where $n\in\NN$, we define
%  $$ \ip{f,g} \defeq \frac1{p^n} \sum_{x\in\bbF^n} f(x)\overline{g(x)}.$$
%The orthonormal basis functions are $\chi_{i_1,\dotsc,i_n}\colon\bbF^n\to\CC$, and 
%  $$\chi_{i_1,\dotsc,i_n}(x_1,\dotsc,x_n) \defeq \exp\left(2\pi \imath \sum_{j=1}^ni_jx_j/p\right), $$
%where $i_1,\dotsc,i_n,x_1,\dotsc,x_n\in\bbF$.
%For $i\in\bbF^n$, we define the Fourier coefficient $\widehat f(i)\defeq \ip{f,\chi_i}$.
%The Parseval's identity states that
%  $$ \ip{f,f} = \sum_{i\in\bbF^n} \widehat f(i)^2.$$

Quoting for completeness, the lemma states
that $\sum_{\ell_i}|\widehat{\vec{1}}_{\ell_i}(\alpha)|= 1$ if $\alpha=0$ and is upper-bounded by the constant $c_m<1$ otherwise (for every constant $m\geq 1$). For instance, $\lim_{p\rightarrow \infty} c_1= 2/\pi$ when $m=1$.

We will need the following technical Lemma from~\cite{EPRINT:BDIR19}.
\begin{lemma}(Lemma 4.17,~\cite{EPRINT:BDIR19}) \label{lem:sumlibound}
Let $\vec{L}\in\mathcal{L}_{m,n,p}$, where $m$ is a constant. Then, for each $i \in [n]$, it holds that 
\begin{align}
\nonumber
\begin{cases}
\sum_{\ell_i}|\widehat{\vec{1}}_{\ell_i}(\alpha)|=1 &\text{if}\alpha=0 \\
\sum_{\ell_i}|\widehat{\vec{1}}_{\ell_i}(\alpha)|\leq c_m &\text{if }\alpha\neq 0\end{cases}
\end{align}
for $c_m=\frac{2^m \sin{(\pi / 2^m)}}{p \sin{(\pi / p)}}$. Furthermore, $c_m=1-\Omega_m(1)$, for sufficiently large $p$.\footnote{For instance, for $\lim_{p\rightarrow \infty} c_1= 2/\pi$.}
\end{lemma}
    





% Improved result for Shamir Secret Sharing and m=1
\chapter{Improved result for Shamir Secret Sharing and m=1}
\label{sec:shamir} 

As in~\cite{EPRINT:BDIR19}, we prove that for every linear $\mdscode{n}{\codedim}{\F_p}$ code $C$, the $(n,t=\codedim)$-secret sharing scheme $Sh_C$ induced by it is leakage resilient for sufficiently large $t$.
We improve from about $t> 0.907\cdot n$ in~\cite{EPRINT:BDIR19} (their full version), to a smaller constant for Shamir and the case of $m=1$ and leakage advantage $\epsilon=2^{-\Omega(n)}$. Our analysis improves the bound for larger constant $m$ as well, but the bound $t=\Omega(n)$ quickly approaches 1 as $m$ grows (similarly to~\cite{EPRINT:BDIR19} analysis, albeit slightly slower).

\begin{theorem}
\label{shamir}
There exists a constant $\epsilon > 0$, so that for every linear $\mdscode{n}{\codedim}{\F_p}$ code $C$, the corresponding Massey $(n,t=\codedim)$-secret sharing scheme $(Sh_C,Rec_C)$ allows for a leakage of a single bit ($m=1$) and error $\leq 2^{-\epsilon n}$, for sufficiently large $n$ and any $t \geq 0.867\cdot n$.
\end{theorem}

Because of space constraints, we shall only present a proof overview for this theorem.
A complete proof is included in Appendix \ref{proof-shamir}.

\noindent{\bfseries Proof overview.} 
We follow a Fourier-analysis based approach. 
We start with a description of~\cite{EPRINT:BDIR19}'s analysis, and explain where our analysis departs from it.
In~\cite{EPRINT:BDIR19} they prove the following upper bound on the leakage advantage achieved by a leakage functions vector $\vec{L}=(L_1,\ldots,L_n)$ for a Massey scheme based on a $\mdscode{n}{\codedim}{\F_p}$ code. 
%is upper bounded (up to a factor of 2) by the following expression, that depends only on the $[n,t,\F_p]$ code $C$. 
Let $H$ denote the generating matrix of $C^\bot$.
Then
\begin{align}\label{eq:start}
\SD{\vec{L}(C)}{\vec{L}(\F^n_p)}&=\sum_{\vec{\ell}\in(\{0,1\}^m)^n}\left|\sum_{\beta\in \F^{\codedim}_p\setminus{\{0\}}}\prod^n_{i=1}{\widehat{\mathbf{1}}}_{\ell_i}(\ip{\beta,h_i})\right|
\end{align}    
%Here $\vec{1}_{\ell_i}$ is an abuse of notation, standing for $\vec{1}_{A_{i,\ell_i}}(x)$ mapping  $\F_p$ to $\{0,1\}$, that outputs 1 if and only if $x\in A_{i,\ell_i}$. Here $A_{i,\ell_i}=\{x\in\F_p|L_i(x)=\ell_i\}$, that is, all inputs on which $L_i$ leaks the value $\ell_i$.

Then, rearrange the sums and take an absolute value of all summands:
\begin{align}\label{eq:intro}
\SD{\mathbf{L}(C)}{\mathbf{L}(\mathbb{F}^n_p)}&\leq \sum_{\beta\in 
\F^{\codedim}_p\setminus{\{0\}}}\prod^n_{i=1}\Big(\sum_{\ell_i}\left|{\widehat{\mathbf{1}}}_{\ell_i}(\ip{\beta,h_i})\right|\Big)\nonumber
\end{align}

Recall that by Observation~\ref{obs:bound}, $2\max_{\mathbf{L}}\SD{\mathbf{L}(C)}{\mathbf{L}(\mathbb{F}^n_p)}$ upper bounds the leakage error of $C$ (but does not necessarily lower-bounds it well).
This step may lose on parameters, but greatly simplifies analysis.\footnote{Indeed, as we demonstrate in Section~\ref{sec:lim}, taking absolute values is very likely suboptimal.} 
To evaluate the above bound, one uses the fact that there are at most $\codedim-1$ zero coordinates $\alpha_i$ in every codeword. Thus, the contribution of some $\beta$ corresponds to
\begin{align}
\Delta_\beta&=\prod_{i\in[n]}\Big(\sum_{\ell_i}\left|\widehat{\vec{1}}_{\ell_i}(\ip{\beta,h_i})\right|\Big)
\end{align}


Since every boolean function's Fourier coefficient $\widehat{f}(0)$ is the largest one (in absolute value), it makes sense to bound the number of such coefficients appearing together.

Then, leaving out many details, they bound the contribution of any individual non-0 coefficients as $\max_{\alpha\neq 0}\widehat{f}(\alpha)$ by some constant $c_1<1$, resulting in a contribution bounded by $c^{\codedim}_1$ of each $\beta$.\footnote{The `max' is enforced by the expression resulting from the Cauchy-Schwartz based expression in the sequel.}
However, bounding every contribution individually turns out to yield very weak bounds on $t=(1-o(1))n$ (even for $m=1$). Cauchy-Schwartz combined with Parseval's identity allows to obtain a better bound. 
Parseval's identity is useful here, as it bounds the $\ell_2$-norm of each $\widehat{\vec{1}}_{\ell_i}$ by $|A_{i,\ell_i}|/p\leq 1$. The final bound takes advantage of the fact that 
$\widehat{f}(0)$ is not much larger than $\widehat{f'}(\alpha)$ for nice boolean functions $f'$.
Then, they replace the $\vec{1}_{\ell_i}$'s involved by such nice functions, in a way that the expression for the bound can only increase, and bound this expression instead. The need for the replacement stems from the fact that nothing is assumed about the locations of the $0$-coefficients in $H\beta^T$ (indeed, these locations vary for different $\beta$'s).

Our improvement here stems by grouping the $\beta$'s by the locations of $H\beta^T$'s 0-coefficients, and bounding the contribution of each group separately. Knowing the locations of 0's allows to use a bound on $\max_{\alpha\neq 0} \widehat{\vec{1}}_{\ell_i}(\alpha)$. %To complete the picture, jumping ahead, our result for random linear MDS codes in Section~\ref{sec:mds}.


% A result for random MDS code over a large fields
\chapter{A result for random MDS code over a large fields}
\label{sec:random-mds} 

Next, we prove that for sufficiently large prime-order fields, `almost all' Massey (n,t)-secret sharing schemes over $\F_{p(n)}$ for sufficiently large $p(n)$, are leakage-resilient for smaller values of $t$ - all the way down to $t\geq (0.5 +\delta)n$, for every constant $\delta>0$. 

We will need a few technical linear-algebraic  claims. The first one is a certain generalization of the rank method used in the communication complexity literature, stating that boolean matrices with $a$ distinct rows have rank at least $\log{a}$ over the reals. We show that over any field, a matrix with $a$ distinct rows where the entries in every column belong to a set of constant size, has rank at least $O(\log{a})$. 

\begin{claim}\label{rank}
Let $M\in\F^{a\times b}_p$ denote a matrix with distinct rows, where $a=2^{cb}$ for some constants 
$c > 0$ and $T\geq 2$. Assume further, that for every $y\in[b]$, there exists a set $V_y\subseteq \F_p$ of size $T$ such that for all $x\in[a]$ and $y\in[b]$ it holds that $M[x,y]\in V_y$. Then $rank(M)\geq  \frac{c}{\log(T)}b$.
\end{claim}

%\TODO{We need to explain how the secret sharing scheme is generated from the code.
%Make sure that all dimensions we use are precise and not off by 1.
%Sould be part of the preliminaries, and briefly mention in the intro as well.}

\begin{proof}
Let $r$ be the rank of $M$. Let $M'=(v_1;\ldots;v_r)$ be a submatrix of $M$, whose rows form a basis for the row space of $M$. Let $I$ denote the (index) set of $r$ independent columns of $M'$. w.l.o.g.\ assume that $I=[r]$. Let $M''=M'[*,I]$. Then, every vector $u\in V_1\times\ldots\times V_b\cap \rowspan{M}$ equals some $h\cdot M'$, where $h\cdot M''\in V_1\times\ldots\times V_r$. As $M''$ is invertible, $h$ is of the form
$h = u[I] {M''}^{-1}$ - that is, $u[I]$ uniquely determines $h$. 
As there are at most $T^r$ such $u[I]$-values, 
\begin{align}
\left|V_1\times\ldots\times V_b \cap \rowspan{M}\right|\leq T^r
\end{align}

So, to generate all (distinct) $a$ rows of $M$, we would need $r\geq \frac{c}{\log(T)}b$.
\end{proof}

Let $C^\bot$ denote a $\mdscode{n}{\dualcodedim}{\F_p}$ generated by the matrix $H=(h_1;\ldots;h_n)$.
For a number $T$ and a vector $\vec{V}=(V_1,\ldots,V_n)\in {\binom{\F_p}{T}}^n$ (each $V_i$ is a set of $T$ field elements), let us denote  $$\Bad{I}{\vec{V}}(C^\bot) = \left\{\beta\in\F^{\dualcodedim}_p| \forall i\in I \text{ and } \ip{\beta,h_i} \in V_i\right\} ,$$ 
and $$\Bad{\vec{V}}{\delta}(C^\bot)=\bigcup_{I\subseteq [n]\text{ of size }(1-\delta) n}\Bad{I}{\vec{V}}(C^\bot).$$
\noindent For a constant $c$ we also denote
$$\Bad{\delta}{c}=\bigcup_{\vec{V}\in{\binom{\F_p}{T}}^n}\left\{C^\bot:\left|\Bad{\vec{V}}{\delta}(C^\bot)\right|\geq c^n\right\}.$$
For a vector $v\in \F^n_p$ we write $v\in\vec{V}$ to mean that $v_i\in V_i$ for all $i \in [n]$. 

The following lemma is a key lemma in our analysis. Roughly, it states that for a small $T$, `most' MDS codes with certain parameters don't have `many' `bad' codeword, such that `many' coordinates out of each bad codeword fall in a set of size $T$ (sets may differ for different coordinates).

\begin{lemma}
\label{lem:sim}
Let $p(n) \geq 2^n$ be a function returning primes, and let $c>1$, $T\geq 2,0<\delta<1/2$ be constants. We further require that $\log{c}>\Ent{\delta}$. 
Consider the set $\mathcal{C}^\bot$ of linear $\mdscode{n}{\dualcodedim}{\F_p}$ codes $C^\bot$, where $\Omega(n)=\dualcodedim\leq (1-2\delta)n$, and the uniform probability distribution over $\mathcal{C}^\bot$. 
%Let $\Bad{\delta}{c}=\bigcup_{\vec{V}\in{\binom{\F_p}{T}}^n}\left\{C^\bot:\left|\Bad{\vec{V}}{\delta}(C^\bot)\right|\geq c^n\right\}$.
Then,
$$Pr_{C^\bot\in \mathcal{C}^\bot}\left[C^\bot \in \Bad{\delta}{c}\right]=neg(n)$$

%We refer to (the few) linear $\mdscode{n}{\dualcodedim}{\F_p}$ code falling in $\Bad{\delta}{c}$ as \emph{bad for} $(n,\delta,c,T)$ ($p$ is determined by $n$).
%In particular, it implies that for a sufficiently large $n$ and $T,c,\delta$ as above, a code $C^\bot$ which is not bad for $(n,\delta,c,T)$ exists.
\end{lemma}

\begin{proof}
Fix a sufficiently large $n$ and $p(n),c,T,\delta,\dualcodedim$ as in the lemma. Consider some linear MDS code $C^\bot\in \Bad{\delta}{c}$. % - in the analysis below, we will in particular show that linear MDS codes with the required parameters do exist.
So, by definition $\Bad{\delta}{c}$, there exists some $\vec{V}$ such that $\Bad{\vec{V}}{\delta}(C^\bot)\geq c^n$. 
By the assumption that $c>\Ent{\delta}$, there exists a set of coordinates $I$ of size $(1-\delta)n$, so that  $\Bad{I}{\vec{V}}(C^\bot)$ is of size $\tilde{\Omega}(2^{(\log(c)-\Ent{\delta})n})$. This follows by a simple averaging argument, and approximating the number of $I$'s of size $(1-\delta)n$ using estimation~\ref{eq:bin} (note that $\Ent{1-\delta}=\Ent{\delta}$). 
Let $c'$ be such that $\log(c')=0.99(\log(c)-\Ent{\delta})$, which is by assumption a positive constant. For each $I$ as above, denote the first $\dualcodedim$ coordinates in $I$ by $I'$. 

Note that for a vector $\vec{u}=H\beta^T$ for $\beta\in\Bad{I}{\vec{V}}(C^\bot)$, $\vec{u}[I']$ uniquely determines a $\beta$. This holds since $C^\bot$ is an MDS code, so $H[I']$ is invertible, and determines $\beta$. Consequently, to determine an element in $\Bad{I}{\vec{V}}(C^\bot)$, it suffices to specify it as a sequence of indices into the set $\vec{V}[I']$, where the set $\vec{V}[I']$ is ordered according to some fixed ordering (say, lexicographically).

We thus sometimes denote elements of $\Bad{I}{\vec{V}}(C^\bot)$ as indices $\vec{b}\in [T]^{\dualcodedim}$, and sometimes explicitly as vectors $\beta\in \F^{\dualcodedim}_p$ determined by them (for a fixed $\vec{V},I$). For an index $\vec{b}$, we denote the (unique) corresponding $\beta$ by $\beta_\vec{b}$, and the vector $\vec{u}\in \vec{V}$ such that $\vec{u}=H\beta^T_\vec{b}$ by $\vec{V}_{I,\vec{b}}$.
From now on, by $\vec{V}$ we implicitly refer to $\vec{V}\in {\binom{\F_p}{T}}^n$ and by $I$ we implicitly denote elements of $\binom{n}{(1-\delta)n}$ (with $I'$ defined based on $I$ as above).

\noindent Our plan consists of two steps:
\begin{enumerate}
\item Fix some $\vec{V}$ and $I$ and $Bad'\subseteq [T]^{\dualcodedim}$ of size ${c'}^n$ 
for some $c'$. Prove the fraction of codes for which $Bad'\subseteq \Bad{I}{\vec{V}}(C^\bot)$ is very small. We refer to such codes $C^\bot$ as bad for $(\vec{V},I,Bad')$. 
\item Then, take a union bound over all possible $\vec{V},I$ and possible choices of $Bad'$ as above.
%Note that we may cut the set of possible $Bad'$s, by considering only those $Bad'\in\binom{[T]^{\dualcodedim}}{c'^n}$, which may occur as a subset of some $\Bad{I}{\vec{V}}(C^\bot)$ using some criteria removing `impossible' $Bad'$s (which we develop later).
\end{enumerate}

 %\footnote{The index-based representation of $\Bad{I}{\vec{V}}(C^\bot)$ has the advantage that having fixed $V$, $\bad{I}{\vec{V}}(C^\bot)$ only determines a subset of it. Naively starting with an arbitrary $\Bad{I}{\vec{V}}(C^\bot)$ would require to chose $c^n$ field elements with no additional restrictions, and would render the union bound meaningless.Even now, some of our $\vec{V},I,\Bad{I}{\vec{V}}(C^\bot)$ may not be consistent with any code with the required parameters, but this level of care would suffice for our proof.}

The above plan would already work as is, but would require an even larger, double exponential, $p(n)$. Instead, we observe that each $C^\bot$ that is bad for some $(\vec{V},I,Bad')$, is also `bad' for $(\vec{V},I,B)$ for some $B\subseteq Bad'$ (which is always the case), where $B$ is much smaller than $Bad'$, and has a certain special property. Instead, we bound in step 1 the fraction of the set of $C^\bot$ bad for $(\vec{V},I,B)$, for $B$'s as above, and take a union bound over these in step 2. Here, we gain in step 2, since the number of triples is much smaller than before. 
As before, the bound for $(\vec{V},I,B)$ we get in step 1 decreases with $p$, but now we can afford making $p(n)$ only single exponential, due to the smaller number of summands in 2.

We proceed to show how $B$ is derived from $Bad'$.
\begin{observation}\label{clm:basis}
Let $(\vec{V},I,Bad')$ where $Bad'\in \binom{[T]^{\dualcodedim}}{c'^n}$ and $C^\bot$ bad for it.  
Then, there exists $B\subseteq Bad'$ of size $r=\theta(n)$, such that
$\{\vec{V}_{I,\vec{d}}|\vec{d}\in B\}$ consists of linearly independent vectors. In particular, $C^\bot$ is bad for $(\vec{V},I,B)$  (by definition of bad for $(\Vec{V},I,D)$ for some $D\subseteq [T]^{k^\bot}$).
\end{observation}

\begin{proof}
We observe that $\big\{\vec{V}_{I,\vec{b}}[I]\big\}_{\vec{b}\in Bad'}$ has rank at least $r=\frac{\log(c')}{\log{T}}n$. The observation follows immediately from applying Claim~\ref{rank} to $M=(H\beta^T_1[I'];\ldots;H\beta^T_{c'^n}[I'])$
where $Bad'=\{\beta^T_1,\ldots,\beta^T_{c'^n}\}$ (indeed, note that $|rows(M)|=2^{|cols(M)|\tilde{c}}$ for some constant $\tilde{c}$, since $cols(M)=\dualcodedim=\Theta(n)$, so the precondition of the claim holds). We set $B$ to be a basis of $M$'s rows.
\end{proof}

Next, following our plan outlined above, we bound the fraction of $C^\bot$'s bad for a given $(\vec{V},I,B)$, where $B$ is as guaranteed by Claim~\ref{clm:basis} for $(\vec{V},I,Bad')$ where $Bad'\in \binom{[T]^{\dualcodedim}}{c'^n}$. 

\begin{claim} \label{clm:probbad}
Let $\vec{V},I,B$ where $B\in \binom{[T]^{\dualcodedim}}{c'^n}$ and $D=\{\vec{V}_{I,\vec{d}}|\vec{d}\in B\}$ consists of linearly independent vectors. Then, 
\begin{align}
Pr_{C^\bot\in\mdscode{n}{\dualcodedim}{\F_p}}\Big[C^\bot\text{ is bad for }(\vec{V},I,B)\Big]=p^{-\Omega(n^2\log{p})}
\end{align}
\end{claim}

\begin{proof}
We sample a uniformly $C^\bot$ by sampling its generating matrix $H$.
As the code should be MDS, every set of $\dualcodedim$ rows of $H$ form a basis for $rows(H)$. In particular, given $I$, $H[I']$ is a basis of its row set. 
For simplicity of notation, we assume w.l.o.g.\ that
$I'=[1,\ldots,\dualcodedim]$. Then to determine the rest of $H[I]$ (which is the part we will be interested in), we should set the variables $\alpha_{i,j}$ in
$$h_i=\sum_{j\in[I']}\alpha_{i,j} h_j$$ 
for each $i\in I \setminus I'$.

\noindent In fact, we do not directly sample the matrix $H[I]$ to be consistent with and MDS code, but rather
sample it according to the following distribution $\mathcal{D}_H$,
\begin{enumerate}
\item First sample $h_1,\ldots,h_{\dualcodedim}$ as random linearly independent vectors.   
\item Sample the $\alpha_{i,j}$s as random independent element of $\F_p$.
\end{enumerate}


\noindent We require that $H$ satisfies the following constraints
\begin{enumerate}
\item Every $\dualcodedim$ rows in the resulting $H[I]$ are linearly independent (to actually obtain $H[I]$ consistent with an MDS code). Let us denote this event by $E_1$. 
\item %Assuming the $\alpha_{i,j}$'s satisfy condition 1, 
Every resulting row $h_i$ is consistent with each $\vec{V}_{I,\vec{d}}$ for each $\vec{d}\in B$. 
Let us denote this event by $E_2$.
\end{enumerate}

Let us explicitly state condition 2.
For each $\vec{d}\in B$, and $i\in I\setminus{I'}$ we have
\begin{align}\label{eq:key} 
\sum_{j\in [I']}\alpha_{i,j}\vec{V}_{I,\vec{d}}[j]=\vec{V}_{I,\vec{d}}[i].
\end{align}

That is, having fixed $H[I']$, $H[I\setminus I']$ satisfies 
a linear equation system of the form
\begin{align}
 M_B\alpha^T_i=v_i \label{eq:key2}
\end{align}
where $M_B\in \F^{r\times \dualcodedim}$ is a full-rank matrix, whose rows are elements in $\big\{\vec{V}_{I,\vec{b}}[I']\big\}_{\vec{b}\in Bad'}$.
We are now ready to prove our theorem - in particular, note that $M_B$ depends only on $\vec{V},I,B$, rather than on the code itself. 
This follows as 
$$\vec{V}_{I,\vec{d}}[i]=H\beta^T_{\vec{d}}[i]=H[i]\beta^T_{\vec{d}}$$
Making the same observation on the left side, together with $h_i=\sum_{j\in [I']}\alpha_{i,j}h_j$ implies Equation~\ref{eq:key}.
We can restate Equation~\ref{eq:key}, as requiring that every $\alpha_i$ satisfies a linear equation system
$M_B\alpha^T_i=\vec{\tilde{u}}_i$, where $M_B$ is invertible (note that $M_B$ is the same for all $i$).

Now we prove that the probability (over a uniform choice of all $\alpha_{i,j}\in \F_p$ and $H[I']$) that constraint 2 holds conditioned on constraint 1 holding, is $p^{-\Omega(\dualcodedim^2\log(p))}$.
That is, we prove
\begin{align}
    Pr_{H\leftarrow \mathacal{D}_H}[C^\bot \in E_2| C^\bot E_1] &\leq \nonumber\\\nonumber \\ 
     \frac{Pr_{H\leftarrow \mathacal{D}_H}[C^\bot \in E_2]}{Pr_{H\leftarrow \mathacal{D}_H}[C^\bot \in E_1]} &= p^{-\Omega(n^2\log{p})} \label{eq:probbound}
\end{align}
To prove Equation~\ref{eq:probbound}, we bound the denominator and the numerator of the expression in~\ref{eq:probbound} separately.

\begin{claim} \label{clm:E2} %%% Claim 4.0.3
$Pr_{H\leftarrow \mathcal{D}_H}[C^\bot \in E_2]=p^{-\Omega(n^2\log{p})}$
\end{claim}

\begin{proof}
Consider having chosen the first $h_1,\ldots,h_{\dualcodedim}$ in $H$ (which are linearly independent). Next we move to picking the rows in $I\setminus I'$, represented in basis $h_1,\ldots,h_{\dualcodedim}$ for convenience (by choosing the $\alpha_{i,j}$s).
Consider the next $i\in I\setminus{I'}$ for which we pick $h_i$.
By properties of linear transformations, and Equation~\ref{eq:key2}, every such coefficient vector $\alpha_i$, belongs to a coset $K+x_i$ of the right kernel $K$ of $M_B$, for some $x_i\in \F^r_p$. %There are at most $T^{(1-\delta)n-\codedim}\leq T^n$ such cosets.
Therefore, a randomly chosen such vector $\alpha_i$ only satisfies the above condition with probability $q = p^{-r}$,
since $|K|=\F^{\dualcodedim-r}$ by the rank-nullity theorem.
Note that this holds independently of the concrete choice of $H[I']$.
Now, $q$ equals $p^{-\Theta(n)}$, since indeed $r=\Theta(n)$ by Observation~\ref{clm:basis}, and $\dualcodedim=\Theta(n)$ by the choice of $\delta$.
As the choice of each $h_i$ is independent of the choice of $h_j$ for every $i,j\in I\setminus{I'}$, the overall probability of the event $E_2$ is 
\begin{align} \label{eq:probE2}
&\prod_{i\in I\setminus{I'}}p^{-r}=p^{-r((1-\delta)n-\dualcodedim)}=p^{-\Theta(n^2)}
\end{align}
\end{proof} %End Claim 4.0.3 Proof

\begin{claim} \label{clm:E1} %%% Claim 4.0.4
$Pr_{H\leftarrow \mathcal{D}_H}[C^\bot \in E_1]>1/2$ assuming $p(n)\geq 2^n$.
\end{claim}

\begin{proof}
Consider the process of randomly choosing the rows in $H[I\setminus{I'}]$, after picking $H[I']$ according to $\mathcal{D}_H$. We pick the rows $h_i$ ($i\in I\setminus{I'}$) one by one. For each row $h_i$ being picked, let us denote the set $\tilde{I}$ of rows picked so far (including $H[I']$ and excluding $h_i$).
We require that the invariant, that every $\dualcodedim\times \dualcodedim$ submatrix of $H[\tilde{I}\cup\{i\}]$ remains invertible, is not broken upon choosing $h_i$. 
To keep the invariant, we need that every submatrix $\tilde{H}$ of $H[\tilde{I}]$ with $\dualcodedim-1$ rows, remains invertible when appending $h_i$ to it.
In the worst case - when choosing the last row, we have at least a fraction 
\begin{align}
1 - {\binom{n}{\dualcodedim}}/p>1/2 \label{eq:frac}
\end{align}
of vectors in $\F^{\dualcodedim}_p$ to choose from. 
This follows by taking a union bound over all submatrices $\tilde{H}$ as above.
Indeed, complementing each $\tilde{H}$ succeeds with probability $1-1/p$. Picking $p(n)$ large enough - $p(n)\geq 2^n$ suffices, keeps Equation~\ref{eq:frac} true.
\end{proof} %End Claim 4.0.4 Proof
Now, Equation~\ref{eq:probbound}, follows immediately from Claim~\ref{clm:E2} and Claim~\ref{clm:E1}, which completes the proof of Claim~\ref{clm:probbad}.
\end{proof} %End Claim 4.0.2 Proof


Back to the proof of Lemma~\ref{lem:sim} (following step 2 of the plan), we take a union bound over all possible 
$\vec{V}\in \binom{\F_p}{T}^n,I\in \binom{[n]}{(1-\delta)n},B\in \binom{[T]^{\dualcodedim}}{r}$. By a crude estimation, there exist at most
$${\binom{p}{T}}^n(2^n)(T^{{\dualcodedim} n})=2^{O((\log{p}+{\dualcodedim}) n)}$$
such triples. Thus, from Claim~\ref{clm:probbad}, the probability of $C^\bot$ being bad for $(n,\delta,c,T)$
is upper bounded by
$$2^{\log{p}(O(n) - \Omega(n^2)) + O(n^2))}=2^{n^2(\Omega(\log{p})+O(1))}=2^{\Omega(-n^3)}=neg(n),$$ 
where the last equality is implied by $\log{p}=\Omega(n)$.
This concludes the proof of Lemma~\ref{lem:sim}.
\end{proof} %End Lemma 4.0.1 Proof (\label{lem:sim})

%As $H[I',*]$ forms a basis of $H$'s row space, $H[I',*]$ is invertible. We may therefore express $H[I\setminus{I'},*]$'s rows in the basis formed by $H[I',*]$'s rows. Assume w.l.o.g.\ that $I'=[1,\ldots,\dualcodedim]$. 
%Then each $h_i$ for $i\in I\setminus{I'}$ satisfies the following equation in the $\alpha_{i,j}$'s: $h_i=\sum_{j\in[I']}\alpha_{i,j} h_j$. Consider some basis $B$ of 
%$\{\vec{V}_{I,\vec{b}}[I]\}_{\vec{b}\in Bad'}$, taken from the set itself - assume it corresponds to indices $b_1,\ldots,b_r$. 

%For each $b_y\in B$, we have
%$\sum_{j\in [I']}\alpha_{i,j}\vec{V}_{I,b_y}[j]=\vec{V}_{I,b_y}[i].$
%That is, having fixed $H[I']$, $H[I\setminus I']$ satisfies 
%a linear equation system of the form
%$M_B\alpha^T_i=v_i$ where $M_B\in \F^{r\times \dualcodedim}$ is a full-rank matrix, whose rows are elements in $\big\{\vec{V}_{I,\vec{b}}[I']\big\}_{\vec{b}\in Bad'}$.
%We are now ready to prove our theorem - in particular, note that $M_B$ depends only on $\vec{V},I,B$, rather than on the code itself. 

%Now, to improve the resulting bound on field size, we update plan 1+2 to consider bad codes for $(\vec{V},I,B)$, and take a union bound on all possible triples. For a given $(\vec{V},I,B)$, we denote the fraction of codes bad for it by $bc_{\vec{V},I,B}$.


\begin{theorem}\label{thm:mostmds}
Let $0<\delta<1$ and $m\in\mathbb{N}^+$ be constants, where $\delta$ is sufficiently small\footnote{The strange situation where we handle small $\delta>0$ but not larger $\delta$ is an artifact of the proof of Lemma~\ref{lem:sim}. A slightly more complicated proof would remove this restriction. See the full version for details.}. Then for every field size function $p(n)\geq 2^n$ outputting primes, every sufficiently large $n$, and every $\codedim \geq (0.5 + \delta/2)n$, the Massey secret  sharing scheme $(Sh_C,Rec_C)$ corresponding to a random linear$ \mdscode{n}{\codedim}{\F_p}$ code $C$, allows for a local leakage of $m$ bits from each party's secret share and leakage error $\leq 2^{-\Omega(n)}$ with overwhelming probability $1-neg(n)$.  
\end{theorem}
% NOTE: I am still in the middle of writing up all the proofs.
% The constant for m=1 is the best, I am yet to calculate it. It is something circa 0.48n

\begin{proof}
\noindent{\bfseries Proof overview.} 
We consider the hardest case of $\codedim=(0.5 + \delta/2)n$. 
Note that since the dual of a linear $\mdscode{n}{\codedim}{\F_p}$ code $C$ is a linear $\mdscode{n}{\dualcodedim = n-\codedim=(0.5-\delta/2)n}{\F_p}$ code $C^\bot$, a random $C$ as above (as considered in the theorem), can be uniformly sampled by uniformly sampling an $\mdscode{n}{\dualcodedim}{\F_p}$ (and taking the dual $C=(C^\bot)^\bot$).
Indeed, it will be more convenient to consider sampling of $C^\bot$ throughout the proof. 
We use the probabilistic method to prove that the $C^\bot$ based expression for $\SD{\vec{L}(C)}{\vec{L}(\F^n_p)}$ from Claim~\ref{clm:dual} is small for `almost all' codes $C^\bot$. This will correspondingly imply the Theorem for almost all codes $C$ as required. 

Our proof proceeds by applying Lemma~\ref{lem:sim} to $\vec{V}$, where each $V_i$ represents a set of values $\alpha_i\in \F_p$ corresponding to Fourier coefficients with  `large' (say $\geq 0.01$) absolute value, of any of the local leakage functions $\widehat{\vec{1}}_{\ell_i}$. 

On a high level, we prove that with overwhelming probability, for a random $\mdscode{n}{\dualcodedim}{\F_p}$ code $C^\bot$, not only that every non-0 codeword has less than $\dualcodedim$ 0-coordinates (which holds for every MDS code), but also, for every vector of boolean functions $(\vec{1}_{\ell_1},\ldots,\vec{1}_{\ell_n})$, very few codewords $(\alpha_1,\ldots,\alpha_n)$ have a `large' number of `large' coefficients (say, coefficients larger than $0.01$).
That is, 0-coefficients $|\widehat{\vec{1}}_{\ell_i}(\alpha_i=0)|$ are large in absolute value, and contribute a lot to the bound in Claim~\ref{clm:dual} if a lot of them `come together' in a single codeword. Indeed the 0-coefficient is the largest in absolute value for all boolean functions $f$. 

However, if possibly smaller but still large coefficients  (including the 0-coefficient, and a few other that are function-specific)
tend to `come together' sufficiently often in a single codeword, this also pushes the bound up, albeit a bit slower. We can afford more such `somewhat heavy' codewords, but not much more. In our previous analysis for Shamir, we gained a little by pinpointing the locations of the 0-coefficients exactly, instead of assuming the worst possible ($\dualcodedim$) number of 0's in every codeword, as done in~\cite{EPRINT:BDIR19}. 

Here we go a step forward, and prove that for almost all $C^\bot$'s as above, for every leakage functions vector $\vec{L} \in {\mathcal{L}}_{m,n,p}$, only few sufficiently `heavy' codewords $H\cdot \beta$ (where `too many' of the coefficients $(\widehat{\vec{1}}_{\ell_1}(\alpha_1),\ldots,\widehat{\vec{1}}_{\ell_n}(\alpha_n))$ are large) exist. Details follow.\\

%Fix some small constant $0<\delta<1$ with an upper bound to be fixed later. 

A key technical observation our proof relies on, is that for any function $f:\F_p\rightarrow \{0,1\}$, there are very few large Fourier coefficients.
More precisely, for constant (independent of $p$) $\epsilon$ let us define $\SBig{f}{\epsilon}=\{\alpha\in\F_p|\widehat{f}(\alpha) \geq \epsilon\}$. Then we have,

\begin{claim}\label{clm:heavy}
Let $p$ be a prime, and let $f:\F_p\rightarrow\{0,1\}$ be a boolean function.
Then, for any $\epsilon\in(0,1]$, $| \SBig{f}{\epsilon}|\leq\epsilon^{-2}$. 
\end{claim} 

\noindent The above simple fact follows immediately from Parseval's identity,
$$\sum_{\alpha\in\F_p}{\widehat{f}}^2(\alpha)=|f^{-1}(1)|/p\leq 1$$

%Next, consider a sufficiently small constant $\epsilon>0$ to be fixed later.
%Let $\delta=log(1/\epsilon)$.

\noindent By Claim~\ref{clm:dual}, we have:
\begin{align}
\SD{\vec{L}(C)}{\vec{L}(\mathbb{F}^n_p)}=\sum_{\vec{\ell}}\left|\sum_{\beta\in 
\F^{\codedim}_p\setminus{\{0\}}}\prod^n_{i=1}\widehat{\vec{1}}_{\ell_i}(\ip{\beta,h_i})\right|.
\end{align}



\noindent Rearranging, we obtain an upper bound
\begin{align}
&\SD{\vec{L}(C)}{\vec{L}(\mathbb{F}^n_p)}\leq \sum_{\beta\in 
\F^{\codedim}_p\setminus{\{0\}}}\sum_{\vec{\ell}}\left|\prod^n_{i=1}\widehat{\vec{1}}_{\ell_i}(\ip{\beta,h_i})\right|\label{eq:8}
\end{align}

%Given a leakage vector $\vec{L}=(L_1,\ldots,L_n) \in {\mathcal{L}}_{m,n,p}$. 
We now consider parameters $c,\delta,\epsilon,T=\lceil 2^m\epsilon^{-2}\rceil$ for $c,\epsilon,\delta$ to be determined later in a way satisfying the conditions of Lemma~\ref{lem:sim}.
Then, a random $\mdscode{n}{\dualcodedim}{\F_p}$ code $C^\bot$ satisfies the condition of Lemma~\ref{lem:sim} with overwhelming probability. Fix any such code $C^\bot$ and let $H$ be its generating matrix.

Consider the sequence $\vec{V}=(V_1,\ldots,V_n)\subseteq {\binom{\F_p}{T}}^n$ where 
$V_i = \bigcup_{\ell_i\in \{0,1\}^m} \SBig{\vec{1}_{\ell_i}}{\epsilon}$ \footnote{Padding $V_i$ to size $T$ arbitrarily} is the set of all values of codeword coordinate $\alpha_i$ corresponding to a large coefficient for some function $\vec{1}_{\ell_i}$. 
\footnote{In particular, the $V_i$'s will always include 0. We could further limit the structure of $V$ in Lemma~\ref{lem:sim}, for instance, that $V_i$ consists of pairs of values of the form $\alpha,-\alpha$, but as it seemingly does not help improve our bounds, we do not.}

%\paragraph{Proof of Lemma~\ref{lem:sim}}
Let $\BadNoZero{\vec{V}}{\delta}=\Bad{\vec{V}}{\delta}(C^\bot) \setminus \{0\}$, $\GoodNoZero{\vec{V}}{\delta}=\F^{\dualcodedim}_p\setminus(\BadNoZero{\vec{V}}{\delta}\cup\{0\})$. For a set $I\subseteq [n]$, we denote \[\GoodNoZero{\delta}{I}=\{\beta\in \GoodNoZero{\vec{V}}{\delta}| \forall i\in I\;(H\beta^T[i]\notin V_i)\}\]
Next, we split the sum in Equation~\ref{eq:8} applied to $C=(C^\bot)^\bot$ as follows, 
\begin{align}
&\SD{\vec{L}(C)}{\vec{L}(\mathbb{F}^n_p)}\leq\nonumber\\
& \sum_{\beta\in \GoodNoZero{\vec{V}}{\delta}}\sum_{\vec{\ell}}\left|\prod^n_{i=1}\widehat{\vec{1}}_{\ell_i}(\ip{\beta,h_i})\right| 
  + \sum_{\beta\in \BadNoZero{\vec{V}}{\delta}}\sum_{\vec{\ell}}\left|\prod^n_{i=1}\widehat{\vec{1}}_{\ell_i}(\ip{\beta,h_i})\right|
\end{align}

Let us bound each of the two summands separately. %Consider the first summand.
%We split the set $Good_{I_1,V}$ into two sets, $\GoodNoZero{\vec{V}}{\delta},\BadNoZero{\vec{V}}{\delta}$. $\GoodNoZero{\vec{v}}{\delta}$ is the set of all vectors
%$\beta$, such that $|\{i\in I_1|\ip{\beta,h_i}\in V_i\}|\geq (1-\delta) n/3$.
%By choice of $C^\bot$, $|\BadNoZero{\vec{V}}{\delta}|\leq 2^{cn}$. 
For a given $I \subseteq [n]$, let us denote by $I_1,I_2$ a partition of $[n]\setminus I$ into two subsets of equal size, in some predetermined way (depending only on $I$). 
We bound the contribution of $\GoodNoZero{\vec{V}}{\delta}$ first,
{\allowdisplaybreaks
\begin{align}
&\sum_{\beta\in \GoodNoZero{\vec{V}}{\delta}}\sum_{\vec{\ell}}\left|\prod^n_{i=1}\widehat{\vec{1}}_{\ell_i}(\ip{\beta,h_i})\right|\leq\nonumber\\
%%% 4.7
&\sum_{I\in{\binom{[n]}{\delta n}[n]}}\sum_{\beta\in \GoodNoZero{\delta}{I}}\sum_{\vec{\ell}}\left|\prod^n_{i=1}\widehat{\vec{1}}_{\ell_i}(\ip{\beta,h_i})\right|\label{eq-CS7} \leq \\
%%% 4.8
&\tilde{O}(2^{\Ent{\delta}n}) \cdot \max_I\Bigg(\sqrt{\sum_{\beta\in \GoodNoZero{\vec{V}}{\delta}}\prod_{i\in I_1}\Big(\sum_{\ell_i}|\widehat{\vec{1}}_{\ell_i}(\ip{\beta,h_i})|^2\Big)}\cdot\sqrt{\sum_{\beta\in \GoodNoZero{\vec{V}}{\delta}}\prod_{i\in I_2}\Big(\sum_{\ell_i}|\widehat{\vec{1}}_{\ell_i}(\ip{\beta,h_i})|^2\Big)}\cdot \nonumber \\ 
%%% 4.8
& \hphantom{{}=\tilde{O}(2^{\Ent{\delta}n})\max_I}\max_{\beta\in \GoodNoZero{\vec{V}}{\delta}}\prod_{i\in I}\Big(\sum_{\ell_i}|\widehat{\vec{1}}_{\ell_i}(\ip{\beta,h_i})|\Big) \Bigg) \leq \label{eq-CS8} \\ 
%====================
&\tilde{O}(2^{\Ent{\delta}n}) \cdot \max_I\Bigg(\sqrt{\sum_{\beta\in \F^{\dualcodedim}_p}\prod_{i\in I_1}\Big(\sum_{\ell_i}|\widehat{\vec{1}}_{\ell_i}(\ip{\beta,h_i})|^2\Big)}\cdot\sqrt{\sum_{\beta\in  \F^{\dualcodedim}_p}\prod_{i\in I_2}\Big(\sum_{\ell_i}|\widehat{\vec{1}}_{\ell_i}(\ip{\beta,h_i})|^2\Big)}\cdot \nonumber\\
%%% 4.9
& \hphantom{{}=\tilde{O}(2^{\Ent{\delta}n})\max_I} \max_{\beta\in \GoodNoZero{\vec{V}}{\delta}}\prod_{i\in I}\Big(\sum_{\ell_i}|\widehat{\vec{1}}_{\ell_i}(\ip{\beta,h_i})|\Big) \Bigg)\leq \label{eq-CS11} \\
&\tilde{O}(2^{\Ent{\delta}n}) \cdot\max_I \Bigg( \sqrt{\prod_{i\in I_1}\Big(\sum_{\ell_i}\sum_{\alpha\in\F_p}|\widehat{\vec{1}}_{\ell_i}(\alpha)|^2\Big)}\cdot\sqrt{\prod_{i\in I_2}\Big(\sum_{\ell_i}\sum_{\alpha\in\F_p}|\widehat{\vec{1}}_{\ell_i}(\alpha)|^2\Big)}\cdot\nonumber\\
%%% 4.10
& \hphantom{{}=\tilde{O}(2^{\Ent{\delta}n})\max_I} \max_{\beta\in \GoodNoZero{\vec{V}}{\delta}}\prod_{i\in I}\Big(\sum_{\ell_i}|\widehat{\vec{1}}_{\ell_i}(\ip{\beta,h_i})|\Big) \Bigg) \leq\label{eq-CS9}\\
%%% 4.11
& \tilde{O}(2^{\Ent{\delta}n})\cdot(2^{2m}\epsilon)^{\delta n}=\tilde{O}(2^{(\Ent{\delta}+\delta(2m+\log{\epsilon}))n})\leq 2^{-\epsilon' n} \label{eq-eps}
\end{align}}

for $\epsilon' > 0$ constant, for a proper choice of $\epsilon$.

Inequality~\ref{eq-CS7} follows by Cauchy–Schwarz.
Let $I$ be the set selected by the  $\max_I$.
Inequality~\ref{eq-CS8} holds since each $\beta\in \F^{\dualcodedim}_p$ contributes a non-negative summand $\prod_{i\in I_1}(\ldots)$ (similarly $I_2$), and all values in the product are non-negative. Thus, adding $\beta$'s beyond $\GoodNoZero{\vec{V}}{\delta}$ can only increase the expression's value.
Inequality~\ref{eq-CS11} holds since $|I_1|,|I_2|=(1-\delta)n/2=(0.5-\delta/2)n$ and this equals exactly $\dualcodedim$. As our code is an MDS code, indeed going over all $\beta\in \F^{\dualcodedim}_p$, contributes exactly 
$$ \prod_{i\in I_1}\Big(\sum_{\alpha\in\F_p}|\widehat{\vec{1}}_{\ell_i}(\alpha)|^2\Big),$$
for each fixed vector of $(\ell_i)_{i\in I_1}$. \footnote{We should indeed make sure that the rounding works out and $(0.5-\delta/2)n$ is integral and even, because otherwise, if $I_1$ is smaller by 1 than $\dualcodedim$, every $\prod_{i\in I_1}\Big(\sum_{\alpha\in\F_p}|\widehat{\vec{1}}_{\ell_i}(\alpha)|^2\Big)$ would appear $p$ times, causing the whole $\sqrt{\prod_{i\in I_1}(\ldots)}$ expressing be multiplied  by $\sqrt{p}$.}
A similar analysis holds for the other $\sqrt{\ldots}$ and $I_2$.

The inequality~\ref{eq-CS9} follows by observing that for every fixed $\ell_i$, $\sum_{\alpha\in\F_p}|\widehat{\vec{1}}_{\ell_i}(\alpha)|^2=\frac{|\vec{1}^{-1}_{\ell_i}(1)|}{p}$ by Parseval's identity, which is upper bounded by $1$. So, summing over all $2^m$ of the $\ell_i$ values results in $2^m$. This implies a $2^{m\delta n}$
bound on the product of the two squares. 
%In the $\max_{\beta\in \GoodNoZero{\vec{V}}{\delta}}$ part, for each $i \in I$ 
%\begin{align}\label{eq:9}
%&\sum_{\ell_i}|\widehat{\vec{1}}_{\ell_i}(\ip{\beta,h_i})|\leq 1
%\end{align}
%This holds since for boolean functions $f$, every coefficient $\widehat{f}(\alpha)$
%contributes is at most as large as the 0-coefficient, which is precisely
%$Pr_{x\in \F_p}[f(x)=1]$. As in our case, the %probabilities over the two values of $\ell_i$ %sum up to exactly 1, the observation follows. 
Now, each $i\in I$ satisfies $\sum_{\ell_i}|\widehat{\vec{1}}_{\ell_i}(\ip{\beta,h_i})|\leq 2^m\epsilon$. This is guaranteed to hold for every $\beta\in \GoodNoZero{\vec{V}}{\delta}$, by the choice of $C^\bot$. Thus, the product of the two contributions is bounded by $2^{m\delta n}(2^m\epsilon)^{\delta n}=(2^{2m}\epsilon)^{\delta n}$.

The inequality~\ref{eq-eps} follows by choosing 
\begin{equation}
\epsilon < 2^{-(\Ent{\delta}/\delta+2m)} \label{epsilonBound},
\end{equation}
so $\epsilon'>0$ is indeed constant,  which concludes the analysis of the bound on $\GoodNoZero{\vec{V}}{\delta}$'s contribution.


The contribution of $\Bad{\setminus\{0\}}{(1-\delta)}$ requires somewhat more care. In particular, we need to show that every fixed $\beta$ makes a contribution which is (much) smaller than 1. We have 
\begin{align}
& \sum_{\beta\in \BadNoZero{\vec{V}}{\delta}}\sum_{\vec{\ell}}\left|\prod^n_{i=1}\widehat{\vec{1}}_{\ell_i}(\ip{\beta,h_i})\right|\leq \nonumber\\
& \sum_{\beta\in \BadNoZero{\vec{V}}{\delta}}\prod^n_{i=1}\Big(\sum_{\ell_i}\left|\widehat{\vec{1}}_{\ell_i}(\ip{\beta,h_i})\right|\Big)\leq\label{eq:applem}\\
& c^n\cdot c^{(0.5 + \delta/2)n}_m = 2^{n\left( \log{c}+(0.5+\delta/2)\log{c_m} \right)}\leq 2^{-\epsilon' n}\label{eq:2epsilon}
\end{align}

for $\epsilon' >0$ constant, for a proper choice of $c$ and $\delta$.

Inequality~\ref{eq:applem} holds since by Lemma~\ref{lem:sim}, we have  $|\Bad{\setminus\{0\}}{(1-\delta)}|\leq c^n$. The second term in the expression $c^n\cdot c^{(0.5 + \delta/2)n}_m$ follows from Lemma~\ref{lem:sumlibound}, applied to each $i\in [n]$, for each fixed $\beta\in \BadNoZero{\vec{V}}{\delta}$. In our case, at most $(0.5-\delta/2)n$ (note that $\dualcodedim-1 <(0.5-\delta/2)n$) coordinates $\ip{\beta,h_i}$ of the codeword $\alpha=H\beta^T$ equal 0, since $\BadNoZero{\vec{V}}{\delta}$ does not contain $\beta = \vec{0}$ and our code $C^\bot$ is MDS. 
Therefore, the contribution of the non-0 coordinates (at least $(0.5+\delta/2)n$ coordinates) to the product accounts to at most $c^{(0.5+\delta/2)n}_m$.
%Now, as every $\beta\in \GoodNoZero{\vec{V}}{\delta}$ is not $0$, there are overall at least $(1/2 + \delta/2)n$ non-zero


The inequality~\ref{eq:2epsilon} and everything so far that relies on Lemma~\ref{lem:sim} follows by choosing 
\begin{align}
    &c \leq c^{-0.5}_m\label{eq:11}\;\;\;\;\;\text{and}\\
    &c \geq 2^{\Ent{\delta}}.\label{eq:12} 
\end{align}

The requirement~\ref{eq:11} suffices to ensure $\epsilon' >0$, since $\delta> 0$. In particular, $c$ satisfying requirement~\ref{eq:11} can be chosen to be a constant strictly larger than 1, since $c_m$ is a constant that falls in $(0,1)$.
The requirement~\ref{eq:12} is to satisfy the precondition $\log(c)>\Ent{\delta}$ of Lemma~\ref{lem:sim}.
%In addition, such a choice of $c$ suffices to sataisfy the preconditions of Lemma~\ref{lem:sim}, since $0<c_m<1$.
%It is easy to see one can set $c,\epsilon$ so that Equations~\ref{eq:11},~\ref{eq:12} and~\ref{eq:9} are satisfied, since $0<c_m<1$. 
To satisfy requirement~\ref{eq:12}, we would need to set $\delta>0$ to be a very small constant, and $\epsilon$ about $2^{-\Theta_m(\delta)}$ by requirement~\ref{epsilonBound}.
This is inconsequential to the scheme's parameters in the current setting. \footnote{It only increases $T$ in Lemma~\ref{lem:sim}, which only possibly affects the smallest $n$ for which a code $C^\bot$ is guaranteed to exist.}
\end{proof}


% Lower bounds
\chapter{Lower bounds}
\label{sec:lower-bounds}

\section{A Lower bound on $t(n)$}

First, we extend the lower bound of Nielsen et al.~\cite{NS20} to the (hardest) setting of 
$m=1$ bits of leakage and arbitrarily large $p$ for the case of linear secret sharing schemes (which have the most applications, as for efficient leakage-resilient MPC). Recall their bound relies on $p$ being relatively small - polynomial in $n$, as originally considered in~\cite{EPRINT:BDIR19}) - on the other hand, their bound is more general in the sense that it is not limited to linear schemes.
More precisely, we have:
%%%
\begin{theorem}\label{thm-lb}
Let $C^+$  be an arbitrary $[n+1,\codedim,\F_p]$ code, where $2\codedim< p, \codedim\leq n$. Let $\mathrm{Sh}_C$ be a Massey secret-sharing scheme corresponding to the linear error-correcting code $C^+$, as described in Section~\ref{sec:prelim}. 
Suppose the secret-sharing scheme $\mathrm{Sh}_C$ is $\epsilon$-local leakage resilient against ${\mathcal L}_{m=1,n,p}$.
Then $\epsilon\geq (\frac{1}{\codedim})^{c\codedim}$, where $c$ is a universal positive constant.
\end{theorem}

The statement in Theorem~\ref{thm-lb} holds for arbitrary $n,p$, and is not asymptotic in nature.
One simple corollary (as $n$ goes to infinity), restating as a bound on $k$.

\begin{corollary}
Let $C^+$  be an arbitrary $[n+1,\codedim,\F_p]$ code, where $2\codedim< p, \codedim\leq n$. Let $\mathrm{Sh}_C$ be a Massey secret-sharing scheme corresponding to the linear error-correcting code $C^+$.
Let ${\mathcal L}_{m,n,p}$ be the set of all functions $\F^n_p\to \left(\zo^m\right)^n$ representing $m$-bit local leakage on each share of the $n$ parties.
Suppose the secret-sharing scheme $\mathrm{Sh}_C$ is $2^{-\Omega(n)}$-local leakage resilient against ${\mathcal L}_{m=1,n,p}$. 
Then $\codedim=\Omega(\frac{n}{\log(n)})$.
\end{corollary}
%%%
The proof of Theorem~\ref{thm-lb} is based on an argument similar to the one made in~\cite{EPRINT:BDIR19}, when proving that $n$ can not be too small, lower bounding the distinguishing advantage attainable for a constant $n$ and additive scheme, by choosing a carefully designed $\vec{L}$.
The extension of the statement and argument are indeed quite simple, and is brought here mostly for completeness.

\begin{proof}
We let $G$ to be the generating matrix of $C^+$. 
Consider the shares given to minimal qualified parties $\shareVec{}[I]$. It is a well known fact about linear secret sharing schemes as above, that
a set of parties can recover the secret iff the rows set of $G[I]$ spans $G[n+1]$. 
As $C^+$ is of dimension $\codedim$, we must have $|I|=\codedim'\leq \codedim$ - assume w.l.o.g.\ that $I=[\codedim']$. Denote $G'=G[I,\cdot]$ and $G''=G[n+1,\cdot]$ (which is a vector). Let $K=Kernel\big(span(\{G''\})\big)$.
Let $\beta\in \F^{\codedim'}_p$ be a vector such that $\beta\cdot G'=G''$. Then, from simple linear algebra, for a random sharing $\shareVec{0}$ of secret $\secret = 0$,
$\shareVec{0}[I]$ is uniform over $C'=\{G'\cdot \vec{k}^T|\vec{k}\in K\}\subseteq \F^{k'}_p$. 
In particular, every $\shareVec{0}[I]$ satisfies

\begin{equation}\label{eq:span1}
\ip{\beta,\shareVec{0}[I]}=\beta^T\cdot G'\vec{k}=G''\vec{k}=0
\end{equation}

Where $\Vec{k}$ is some element of $K$. Let $i\in [\codedim']$ where $\beta_i\neq 0$ (it exists, as $G''\neq 0$).
Rewriting Equation~\ref{eq:span1}, we get
\begin{equation}\label{eq:span2}
\shareVec{0}[i]=-\sum_{j\in [\codedim']\setminus{\{i\}}}\frac{\beta_j}{\beta_i}\shareVec{0}[j]
\end{equation}
Let $h_0=\lfloor p/(2\codedim')\rfloor$. For each $j\in [\codedim']\setminus{\{i\}}$, if $\beta_j\neq 0$, set $A_j=\{\frac{-\beta_i}{\beta_j}a|a\in\{0,\ldots,h_0-1\}\}$. 
Otherwise, set $A_j=\{0,\ldots,h_0-1\}$ (in fact, the latter can be an arbitrary sufficiently large set). For $i$ let $A_i=\{0,\ldots,(p-1)/2\}$.
Let $\vec{L}=(L_1,\ldots,L_n)$ where $L_n(\shareVec{0}[i])=1$ if $sh_i\in A_i$, and $L_n(\shareVec{0}[i])=0$ otherwise. 
Now, we observe
\begin{claim}\label{clm-ajprob}
$Pr[\shareVec{0}[I]\in A_1\times \ldots\times A_{k'}]=\Pi_{j\in [k']\setminus{\{i\}}}|A_j|/p$
\end{claim}

\begin{proof}
To see this, we observe that for a random $x\in C'$, $x[I\setminus{i}]$ is uniform over $\F^{\codedim'-1}_p$.
Then, by definition of the $A_j$'s, the probability of
\begin{align}
&Pr\left[ \shareVec{0}[I\setminus\{i\}]\in A_1\times \ldots\times A_{i-1}\times A_{i+1}\ldots A_{k'} \right] = \Pi_{j\in [k']\setminus{\{i\}}}|A_j|/p \nonumber.
\end{align}

Now, conditioned on the event - $\shareVec{0}[I\setminus\{i\}]\in A_1\times \ldots\times A_{i-1}\times A_{i+1}\ldots A_{k'}$, $\shareVec{0}[i]\in A_i$ by Equation~\ref{eq:span2}, and the choice of the other $A_j$s.
Namely - with probability 1, $\shareVec{0}[i]$ is the sum of at most $\codedim'\leq \codedim$ elements, each in $\{0,\ldots,h_0-1\}$, and the result follows.

To prove that $x[I\setminus{i}]$ is uniform over $\F^{\codedim'-1}_p$, it suffices to prove $C'[I\setminus{i}]$ is indeed of dimension $\codedim'-1$.
As $P_I$ is a minterm of $Sh_C$, all rows in $G'$ are linearly independent. Furthermore, as $K$ is the right kernel of $span(G'')$, $span(G'')$ is the entire left kernel of $K$. Assuming for contradiction that $rowspan(G[I\setminus{\{i\}},\cdot])$ has dimension $<\codedim'-1$, this implies the existence of a vector $\beta'$ with $\beta'[i]=0$, such that $\beta'\cdot G''$ is in the left kernel of $K$, and thus a multiple of 
$G''$, contradicting the fact that $P_I$ is a minterm.
\end{proof}

Substituting the sizes of the $|A_j|$'s in Claim~\ref{clm-ajprob}, we have
\begin{align}\label{eq:s0}
Pr\Big[\shareVec{0}[I]\in A_1\times \ldots\times A_{\codedim'}\Big] &\geq \Big(\lfloor p/(2k')\rfloor/p\Big)^{\codedim-1} \\ 
&\geq \Big(1/(2\codedim)-(1-1/2\codedim)/p\Big)^{\codedim-1} \nonumber\\
&\geq \Big(1/\codedim(2\codedim+1)\Big)^{\codedim}\geq 1/(3\codedim)^{2k} \nonumber
\end{align}

Here the second transition is due to rounding issues, and the one before last transition uses the fact that $p\geq 2\codedim+1$.\footnote{Taking the bound on $p$ to be slightly larger, say $p\geq 3\codedim$ would make the bound on the error about $(1/\Omega(\codedim))^\codedim$, but this is not very significant.} 

On the other hand, for a random sharing of a random secret $\secret \in \F_p$, the distribution of 
$\shareVec{}[I]$ is uniform over $\F^{\codedim'}_p$ (as now the randomness vector used by the sharing scheme is picked at random from $\F^\codedim_p$), and $G'$ is of full rank. Thus, 

\begin{align}
 Pr\Big[\shareVec{}[I]\in A_1\times \ldots\times A_{\codedim'} \Big] &= Pr\Big[\shareVec{0}[I]\in A_1\times \ldots\times A_{\codedim'}\Big]|A_i|/p \nonumber \\
 &\leq 0.5 Pr\Big[\shareVec{0}[I]\in A_1\times \ldots\times A_{\codedim'}\Big] \nonumber
\end{align}


Thus, from Equation~\ref{eq:s0} we have $Pr\big[\vec{L}(\shareVec{0})[I]=\vec{1}\big]-Pr\big[\vec{L}(\shareVec{})[I]=\vec{1}\big]\geq 0.5/(3\codedim)^{2\codedim}$. 
Therefor, by an averaging argument, there exists a secret $\secret_1$, such that 
$\SD{\shareVec{\secret_1}[I]}{\shareVec{0}[I]}\geq 0.5/(3\codedim)^{2\codedim}$. Therefore, 

\begin{align}
\SD{\vec{L}(\shareVec{\secret_1})[I]}{\vec{L}(\shareVec{0})[I]} &\geq 0.5\Big(Pr\big[\vec{L}(\shareVec{0})[I]=
\vec{1}\big]-Pr\big[\vec{L}(\shareVec{\secret_1})[I]=\vec{1}\big]\Big) \nonumber \\
& \geq 0.25/(3\codedim)^{2\codedim} \nonumber
\end{align}\end{proof}


\begin{remark}
In fact, it follows from the proof that the lower bound is slightly stronger then stated, 
if a minterm of size $\codedim'\leq \codedim$ exist. This is possible only in non-MDS codes. 
\end{remark}

\section{Limitations of our techniques}\label{sec:lim}

In our proofs so far, both for all MDS codes, and an average MDS code, we relied on bounding the magnitudes of the Fourier coefficient products in the expression for the bound in Equation~\ref{eq:start}.
We demonstrate that in this case, improving over Cauchy-Schwarz while still taking absolute values of Fourier coefficients can not yield $2^{-\Omega(n)}$ error upper bounds, in case we only use Parseval's identity to classify the set of possible Fourier distributions of the boolean leakage functions.\footnote{The larger culprit appears to be taking absolute values - we believe that a counterexample similar to the hypothetical one below can be demonstrated on real Fourier spectrums of boolean leakage functions.} 

\paragraph{An example illustrating that new techniques are required.}
Consider a hypothetical function vector $\mathbf{L}=(L_1,\ldots,L_n) \in {\mathcal L}_{m,n,p}$, where all
$A_{i,\ell_i}$'s corresponding to some leakage vector $\ell$, are of size $(p\pm 1)/2$ from an MDS code $C$ with parameters $[n,\codedim,\F_p]$ (corresponding to sharings of 0), where for every $i,\ell_i,\alpha_i\neq 0$ the coefficient $\widehat{\vec{1}}_{\ell_i}(\alpha_i)$ is of magnitude $1/\sqrt{2p}$ (the 0 coefficient is $1/2$). 
Such a function vector does not necessarily exist, but
it is consistent with Parseval's identity (more precisely, the coefficients should be $\sqrt{(p\pm 1)/2p^{3/2}}$ to be consistent, but this is inconsequential for our purposes).
Replacing all coefficients with their absolute values and substituting into 
Equation~\ref{eq:start}, we would get a bound of 

\begin{align}
\SD{\vec{L}(C)}{\vec{L}(\F^n_p)} &=\sum_{\vec{\ell}} \left|\sum_{\alpha\in C^{\bot}\setminus{\{0\}}}\prod^n_{i=1}\widehat{\mathbf{1}}_{\ell_i}(\alpha_i) \right|\nonumber \\ \label{eq:upbound}
& \leq 2^m \cdot \left|\sum_{\alpha\in C^{\bot}\setminus{\{0\}}}\prod^n_{i=1}\widehat{\mathbf{1}}_{\ell_i=0}(\alpha_i) \right| \\
&\leq 2^m \cdot \prod_{i\in[n-\codedim]}\Big(\sum_{\alpha_i\in\F_p}|\widehat{\mathbf{1}}_{\ell_i=0}(\alpha_i)|\Big)\Big(\frac{1}{\sqrt{2p}}\Big)^{\codedim} \label{eq:bound} \\\nonumber
&= 2^m \cdot \Big(\frac{1}{\sqrt{2p}}\cdot p\Big)^{n-\codedim}\Big(\frac{1}{\sqrt{2p}}\Big)^\codedim\\
&= 2^m \cdot \Big(\frac{1}{2}\Big)^{n/2}\sqrt{p}^{n-2\codedim}\nonumber
\end{align}

In Equation~\ref{eq:upbound}, we assumes w.l.o.g.\ that $\ell=\vec{0}$ yields the largest $\left|\sum_{\alpha\in C^{\bot}\setminus{\{0\}}}\prod^n_{i=1}\widehat{\mathbf{1}}_{\ell_i=0}(\alpha_i) \right|$.

In~\ref{eq:bound} we bound 
the expression in~\ref{eq:upbound} using the current techniques - that is, bound each $\widehat{\mathbf{1}}_{\ell_i=0}(\alpha_i)$ by its absolute value and assume the Fourier distribution of $\vec{1}_{\ell_i=0}$ is as above from here and on.
Also, to simplify matters, we did not subtract the coefficient product corresponding to the 0-codeword, as it does not significantly increase the resulting upper bound and replaced it by $\frac{1}{\sqrt{2p}}$.

Note that indeed, for large enough $p$ ($p\gg2^n$), the above bound can be made arbitrarily large.

This is clearly not tight, as \emph{all} leakage vectors' contributions together sum up to 1, as these are expressions for a statistical distance between a certain pair of probability distributions. 
Note that looking at a ``slightly more realistic'' Fourier spectrum, where the $\widehat{\vec{1}}_{\ell_i}(0)$'s are the correct $|A_{i,\ell_i}|/p$ values, the above bound is not improved, and even grows somewhat. 

\begin{remark}
A simple calculation reveals that taking absolute values of Fourier coefficients as above, yields
similar lower bounds for the true statistical distance between the leakage from sharings of any pair of secrets $s_1,s_2$ (rather than upper bounding it by a statistical distance between leakage from a sharing of $s=0$ and leakage from a uniform distribution over $\F^n_p$, as we currently do).
\end{remark}

We conclude that to get a true estimation of the leakage, additional ideas will be needed to get a more precise Fourier Analysis (or use a different approach altogether).


%----------------------------------------------------------------------------------------
%	THESIS CONTENT - APPENDICES
%----------------------------------------------------------------------------------------

\appendix % Cue to tell LaTeX that the following "chapters" are Appendices

% Include the appendices of the thesis as separate files from the Appendices folder
% Uncomment the lines as you write the Appendices

% Appendix Template
\chapter{Appendix}

\section{Proof of Theorem \ref{shamir}} % Main appendix title
\label{proof-shamir}
%\label{AppendixX} % Change X to a consecutive letter; for referencing this appendix elsewhere, use \ref{AppendixX}

Let $C$ be a linear $\mdscode{n+1}{\codedim}{\F_p}$ code. The scheme induces a $(n,t=\codedim)$ linear secret sharing scheme $Sh_C$ as explained in Section~\ref{sec:prelim}.
As mentioned in Section~\ref{sec:prelim}, the leakage advantage of the scheme is upper bounded by $\SD{\vec{L}(C)}{\vec{L}(\F^n_p)}$, which we bound next. 

We start as in~\cite{EPRINT:BDIR19}, in the proof of their Lemma 4.18. These (identical)
calculations appear in the first three lines of the derivation below. Here $I_1,I_2$ are disjoint sets of coordinates of size $n-t+1$ each, and $I_3=[n]\setminus{(I_1\cup I_2)}$. For $\beta\in 
\F^{n-t+1}_p\setminus{\{0\}}$ we define $K_j=\{k' \in I_j | \ip{\beta,h_{k'}}=0 \}$.
%, and for $I'_j \subseteq I_j$ we define $K'_j=\{k \in I'_j | \ip{\beta,h_k}=0 \}$.

{\allowdisplaybreaks
\begin{align}
%%% A.1
&\SD{\vec{L}(C)}{\vec{L}(\F^n_p)}=\sum_{\vec{\ell}} \left|\sum_{\alpha\in C^{\bot}\setminus{\{0\}}}\prod^n_{i=1}\widehat{\mathbf{1}}_{\ell_i}(\alpha_i) \right|\label{eq:2}\\ &\nonumber \\
%%% A.2
&=\sum_{\vec{\ell}}\left|\sum_{\beta\in \F^{n-t+1}_p\setminus{\{0\}}}\prod^n_{i=1}\widehat{\mathbf{1}}_{\ell_i}(\ip{\beta,h_i})\right|\label{eq:3}\\ &\nonumber \\
%%% A.3  
&= \sum_{\vec{\ell}}\left|\sum_{\beta\in  \F^{n-t+1}_p\setminus{\{0\}}}\left(\prod_{i\in I_1}\widehat{\mathbf{1}}_{\ell_i}(\ip{\beta,h_i})\right)\cdot \left(\prod_{i\in I_2}\widehat{\mathbf{1}}_{\ell_i}(\ip{\beta,h_i})\right)\cdot \left(\prod_{i\in I_3}\widehat{\mathbf{1}}_{\ell_i}(\ip{\beta,h_i})\right)\right|\label{eq:4}\\ &\nonumber \\
%%%18
%&\leq \sum_{\vec{\ell}}\sum_{0\leq i+j+l\leq n-t+1} \\
%&\hphantom{{}=\leq \sum_{\vec{\ell}}}\sum_{\substack{\beta\in \F^{n-t+1}_p\setminus{\{0\}}\label{eq:5} s.t. \\ |K_1|=i,\\ |K_2|=j,\\ |K_3|=l}} \left|\left(\prod_{x\in I_1}\hat{\mathbf{1}_{\ell_x}}(\ip{\beta,h_x})\right)\cdot \left(\prod_{y\in I_2}\hat{\mathbf{1}_{\ell_y}}(\ip{\beta,h_y})\right)\cdot \left(\prod_{z\in I_3}\hat{\mathbf{1}_{\ell_z}}(\ip{\beta,h_z})\right) \right| \nonumber \\ &\nonumber \\
%%% A.4
&\leq \sum_{\vec{\ell}}\sum_{0\leq i+j+l\leq n-t+1} \; \sum_{\substack{\beta\in \F^{n-t+1}_p\setminus{\{0\}}\label{eq:5} \\ s.t. \; |K_1|=i,\\ \indent |K_2|=j,\\ \indent |K_3|=l}} \Bigg|\left(\prod_{x\in I_1}\widehat{\mathbf{1}}_{\ell_x}(\ip{\beta,h_x})\right)\cdot \\
& \hphantom{{}=\sum_{\vec{\ell}}\sum_{0\leq i+j+l\leq n-t+1} \sum_{\substack{\beta\in \F^{n-t+1}_p\setminus{\{0\}} s.t. \\ |K_1|=i,\\ |K_2|=j,\\ |K_3|=l}}} \left(\prod_{y\in I_2}\widehat{\mathbf{1}}_{\ell_y}(\ip{\beta,h_y})\right)\cdot \left(\prod_{z\in I_3}\widehat{\mathbf{1}}_{\ell_z}(\ip{\beta,h_z})\right) \Bigg| \nonumber \\ &\nonumber \\
%%%19
%&= \sum_{\vec{\ell}}\sum_{0\leq i+j+l\leq n-t+1} \sum_{\substack{I'_1\subseteq I_1,|I'_1|=i\\ \;\;\;\;I'_2\subseteq I_2,|I'_2|=j\\ \;\;\;\;I'_3\subseteq I_3,|I'_3|=l\\}}\\ &\hphantom{{}==\sum_{\vec{\ell}}\sum_{0\leq i+j+l\leq n-t+1}} \sum_{\substack{\beta\in \F^{n-t+1}_p\setminus{\{0\}}\label{eq:6} s.t.\\\forall k\in K'_1\\ \forall k\in K'_2\\ \forall k\in K'_3}} \left|\left(\prod_{x\in I_1}\hat{\mathbf{1}_{\ell_x}}(\ip{\beta,h_x})\right)\cdot \left(\prod_{y\in I_2}\hat{\mathbf{1}_{\ell_y}}(\ip{\beta,h_y})\right)\cdot \left(\prod_{z\in I_3}\hat{\mathbf{1}_{\ell_z}}(\ip{\beta,h_z})\right)\right|\nonumber \\ &\nonumber \\
%%% A.5
&= \sum_{\vec{\ell}}\sum_{0\leq i+j+l\leq n-t+1} \; \sum_{\substack{I'_1\subseteq I_1,|I'_1|=i\\ I'_2\subseteq I_2,|I'_2|=j\\ I'_3\subseteq I_3,|I'_3|=l\\}} \; \sum_{\substack{\beta\in \F^{n-t+1}_p\setminus{\{0\}}\label{eq:6}\\ s.t.\;K_1=I'_1\\ \indent K_2=I'_2\\ \indent K_3=I'_3}} 
\Bigg|
\left(\prod_{x\in I_1}\widehat{\mathbf{1}}_{\ell_x}(\ip{\beta,h_x})\right)\cdot \\
& \hphantom{{}=\sum_{\vec{\ell}}\sum_{0\leq i+j+l\leq n-t+1} \sum_{\substack{I'_1\subseteq I_1,|I'_1|=i\\ I'_2\subseteq I_2,|I'_2|=j\\ I'_3\subseteq I_3,|I'_3|=l\\}} \sum_{\substack{\beta\in \F^{n-t+1}_p\setminus{\{0\}} s.t.\\ K_1=I'_1\\ K_2=I'_2\\ K_3=I'_3}}} 
\left(\prod_{y\in I_2}\widehat{\mathbf{1}}_{\ell_y}(\ip{\beta,h_y})\right)\cdot 
\left(\prod_{z\in I_3}\widehat{\mathbf{1}}_{\ell_z}(\ip{\beta,h_z})\right)
\Bigg|\nonumber \\ 
&\nonumber \\
%%% A.6
&= \sum_{\vec{\ell}}\sum_{0\leq i+j+l\leq n-t+1} \; \sum_{\substack{I'_1\subseteq I_1,|I'_1|=i\\ I'_2\subseteq I_2,|I'_2|=j\\ I'_3\subseteq I_3,|I'_3|=l\\}}S_{I'_1,I'_2,I'_3}
%%%21
\end{align}}

Recall $H$ is a matrix who's columns span $C^\bot$, and let $h_j$ denote the $j$'s row of $H$.
Similarly to~\cite{EPRINT:BDIR19}, the next step is to bound the last expression in the chain using the Cauchy-Schwarz inequality.
The main innovation we introduce is splitting the sum in the expression into several sums as in Equation~\ref{eq:5}, and applying the bound separately to each sum.
Note the implicit definition of $S_{I'_1,I'_2,I'_3}$ in the last line.

\noindent Let us denote 
%\[B_{i,j,l}=\{\beta\in C^\bot\setminus{\{0\}}|\{k\in I_1|\ip{\beta,h_k}=0\}=i, |\{k\in I_2|\ip{\beta,h_k}=0\}|=j,|\{k\in I_3|\ip{\beta,h_k}=0\}|=l\}.\] 
$$B_{i,j,l}=\big\{\beta\in  \F^{n-t+1}_p\setminus{\{0\}} \Big| \left| K_1 \right|=i, \left| K_2 \right|=j, \left| K_3 \right|=l\big\}.$$  
\noindent We further denote 
% $B_{I'_1,I'_2,I'_3}=\{\beta\in B_{i,j,l} \Big| \forall k\in [3]\{u\in I_k|\ip{\beta,h_u}=0\}=I'_k\}$.
$$B_{I'_1,I'_2,I'_3}=\big\{\beta\in B_{i,j,l} \Big| I'_1=K_1, I'_2=K_2, I'_3=K_3 \big\}. $$

%We denote the subset $B'_{I'_1,I'_2,I'_3}$ of %$B_{I'_1,I'_2,I'_3}$ where $\bigcup_{i} I'_i$ is the \emph{exact} 

\noindent Rewriting the above inequality, we get

{\allowdisplaybreaks
\begin{align}
&\SD{\vec{L}(C)}{\vec{L}(\F^n_p)} \nonumber\\
%%% A.7
&\leq \sum_{\vec{\ell}} \sum_{0\leq i+j+l\leq n-t+1} \; \sum_{ \substack{\beta\in B_{i,j,l} \\ s.t.\; |K_1|=i,\\ \indent |K_2|=j,\\ \indent |K_3|=l}} \left| \left( \prod_{x\in I_1}\widehat{\mathbf{1}}_{\ell_x}(\ip{\beta,h_x}) \right) \cdot \left( \prod_{y\in I_2}\widehat{\mathbf{1}}_{\ell_y}(\ip{\beta,h_y}) \right)	\right| \cdot \label{eq-5} \\
&\hphantom{{}=\leq \sum_{\vec{\ell}} \sum_{0\leq i+j+l\leq n-t+1} \sum_{	\substack{\beta\in B_{i,j,l} s.t.\\ |K_1|=i,\\ |K_2|=j,\\|K_3|=l}}}\max_{\beta\in B_{i,j,l}} \left| \prod_{z\in I_3} \widehat{\mathbf{1}}_{\ell_z}(\ip{\beta,h_z}) \right| \nonumber \\ &\nonumber \\
%%% A.8
&\leq \sum_{\vec{\ell}} \sum_{0\leq i+j+l\leq n-t+1} \; \sum_{\substack{I'_1\subseteq I_1,|I'_1|=i, \\I'_2\subseteq I_2,|I'_2|=j, \\I'_3\subseteq I_3,|I'_3|=l}} \sqrt{\sum_{\substack{\beta\in B_{I'_1,I'_2,I'_3}}} \prod_{x\in I_1}|\widehat{\mathbf{1}}_{\ell_x}(\ip{\beta,h_x})|^2}\;\;\cdot \label{eq:7} \\
&\hphantom{{}=\leq \sum_{\vec{\ell}} \sum_{0\leq i+j+l\leq n-t+1} \sum_{\substack{I'_1\subseteq I_1,|I'_1|=i,\\I'_2\subseteq I_2,|I'_2|=j,\\I'_3\subseteq I_3,|I'_3|=l}}} \sqrt{\sum_{\substack{\beta\in B_{I'_1,I'_2,I'_3}}} \prod_{y\in I_2}|\widehat{\mathbf{1}}_{\ell_y}(\ip{\beta,h_y})|^2} \cdot\max_{\beta\in B_{i,j,l}}\left|\prod_{z\in I_3}\widehat{\mathbf{1}}_{\ell_z}(\ip{\beta,h_z})\right| \nonumber
%%%23
\end{align}}

The inequality~\ref{eq-5} is simply due to splitting the sum according to the \emph{exact} locations of $0$ coordinates in $H\beta^T$.
The inequality~\ref{eq:7} is by Cauchy-Schwartz, where the maximum is taken over a larger set than needed, which clearly only increases the bound.

To continue, we will use the following technical observation.

\begin{observation}
\label{clm:tech}
For sets $A_1,A_2$ with $|A_1|+|A_2|=p$, we have $\sum\limits_{i\in[2]}\sqrt{\sum\limits_{\alpha\neq 0}|\widehat{\vec{1}}_{A_i}(\alpha)|^2}\leq 1$.
\end{observation}

\begin{proof}
By Parseval's identity, we have
$||\widehat{\vec{1}}_{A_i}||^2_2 = ||\vec{1}_{A_i}||^2_2=\frac{|A_i|}{p}$. 
Let us denote $q_i=\frac{|A_i|}{p}$.
Thus
%\[||1_{A_i}||_2=\sqrt{\sum_{\alpha\neq 0}|\hat{A_i}(\alpha)|^2} = \sqrt{q_i - q^2_i},\]%
$$\sum_{\alpha\neq 0}|\widehat{\vec{1}}_{A_i}(\alpha)|^2 = q_i - q^2_i$$
as $|\widehat{\vec{1}}_{A_i}(0)|=q_i$. Thus, we have
\begin{equation}
%%% A.9
\label{eq:1}
\sum^2_{i=1}\sqrt{\sum_{\alpha\neq 0}|\widehat{\vec{1}}_{A_i}(\alpha)|^2} = \sqrt{q_1 - q^2_1} + \sqrt{q_2 - q^2_2}
\end{equation}
We denote $g(x)=\sqrt{x(1-x)}$ and observe that the expression in Equation~\ref{eq:1} equals $g(q_1)+g(1-q_1)=2g(q_1)$ (as $q_2=(1-q_1)$ and $g(q) = g(1-q)$).
The maximum of $2g(q_1)$ in $[0,1]$ is easily seen to be obtained at $q_1=0.5$, and equal $1$.
\footnote{Part of our gain comes from a more careful analysis for the case of two sets. Here we compute the maximum exactly instead of using concavity arguments to bound it, as is done in~\cite{EPRINT:BDIR19}. This is not surprising, because they did not attempt to optimize bounds for the case of $m=1$.}
\end{proof}
Let us bound the contribution of the portion a single $I'_1,I'_2,I'_3$ contributed to the expression
in Equation~\ref{eq:7} - 
we denote this contribution by $S'_{I'_1,I'_2,I'_3}$. 

{\allowdisplaybreaks
\begin{align}
&S'_{I'_1,I'_2,I'_3}\nonumber\\ \nonumber \\
%%% A.10
&=\sum_{\vec{\ell}} \sqrt{\sum_{\substack{\beta\in B_{I'_1,I'_2,I'_3}}} \prod_{x\in I_1}|\widehat{\mathbf{1}}_{\ell_x}(\ip{\beta,h_x})|^2}\cdot \sqrt{\sum_{\substack{\beta\in B_{I'_1,I'_2,I'_3}}} \prod_{y\in I_2}|\widehat{\mathbf{1}}_{\ell_y}(\ip{\beta,h_y})|^2} \cdot \label{eq-8} \\ \nonumber
&\hphantom{{}=\sum_{\vec{\ell}}} \max_{\beta\in B_{i,j,l}}\left| \prod_{z\in I_3}\widehat{\mathbf{1}}_{\ell_z}(\ip{\beta,h_z})\right| \\ \nonumber \\
%%% A.11
&\leq \sum_{\vec{\ell}} \sqrt{ \prod_{x\in I'_1}(\frac{|A_x|}{p})^2 \cdot \prod_{x\in \freeione}\left(\sum_{\alpha\in\F_p\setminus{\{0\}}} |\widehat{\mathbf{1}}_{\ell_x}(\alpha)|^2\right)\cdot \prod_{x\in I_1\setminus{(\freeione\cup I'_1})} \max_{\alpha\in\F_p\setminus{\{0\}}}|\widehat{\mathbf{1}}_{\ell_x}(\alpha)|^2 \label{eq-9}}\cdot\\\nonumber
&\hphantom{{}=\sum_{\vec{\ell}}}\sqrt{ \prod_{y\in I'_2}(\frac{|A_y|}{p})^2 \cdot \prod_{y\in \freeitwo}\left(\sum_{\alpha\in\F_p\setminus{\{0\}}} |\widehat{\mathbf{1}}_{\ell_y}(\alpha)|^2\right)\cdot \prod_{y\in I_2\setminus{(\freeitwo\cup I'_2})} \max_{\alpha\in\F_p\setminus{\{0\}}}|\widehat{\mathbf{1}}_{\ell_y}(\alpha)|^2\label{eq-10} }\cdot\\\nonumber
&\hphantom{{}=\sum_{\vec{\ell}}}\prod_{z\in I'_3}(\frac{|A_z|}{p}) \cdot \prod_{z\in I_3\setminus{I'_3}}|\max_{\alpha\in \F_p\setminus{\{0\}}}\widehat{\mathbf{1}}_{\ell_z}(\alpha)|
\end{align}}

\noindent Therefore we have
\begin{align} \label{eq:A.12}
%%% A.12
\SD{\vec{L}(C)}{\vec{L}(\F^n_p)} \leq \sum_{\substack{(I'_1,I'_2,I'_3)\\ s.t. \; \forall j \in [3] \; I'_j\subseteq I_j, \\ \;\;\;\; | \bigcup_{j \in [3]} I'_j | \leq n-t+1}} S'_{I'_1,I'_2,I'_3}
%|\bigcup\limits_{j \in [3]} I'_j|\leq n-t+1}S'_{I'_1,I'_2,I'_3}
%|I'_1 \cup I'_2 \cup I'_3|
\end{align}

We start by separately bounding each of the sums $S'_{I'_1,I'_2,I'_3}$.

\noindent We define the notation $\freeind{\cdot}{\cdot}{\cdot}{\cdot}$ and explain the above sequence of inequalities next. 

For $I\subseteq [n]$ satisfying $|I|\geq n-t+1$ and $I'_1,I'_2,I'_3$
with $|\cup_{j\in [3]}I'_j|\leq n-t+1$, we let $\freeind{I}{I'_1}{I'_2}{I'_3}$ denote a subset of $I\setminus{\cup_{j\in [3]}I'_j}$ of size $n-t+1-|\cup_{j\in [3]}I'_j|$ (picked arbitrarily, say according to lexicographic order). When $I'_1,I'_2,I'_3$ are clear from the context, we abbreviate $\freei=\freeind{I_i}{I'_1}{I'_2}{I'_3}$.

Consider a superset $B'_{I'_1,I'_2,I'_3}$ of $B_{I'_1,I'_2,I'_3}$ consisting of all $\beta\in \F^{n-t+1}_p$ such that
$H\beta^T$ has 0's at all coordinates $\bigcup_{j \in [3]} I'_j$, but may as well have $0$'s \emph{elsewhere}.
Restricting $\beta$ to $B'_{I'_1,I'_2,I'_3}$
results in a linear code $B'=\{H\beta^T|\beta\in B'_{I'_1,I'_2,I'_3}\}$.
From now on, for a set $B\subseteq \F^{n-t+1}_p$ we abbreviate the multi-set 
$\{H\beta^T|\beta\in B\}$ by $H\cdot B$.
It is well known that $C$ corresponding to $(n,t)$-Shamir secret sharing scheme (which is a special case of a Massey code) is $\mdscode{n}{\codedim=t-1}{\F_p}$ code. Therefore, $C^\bot$ 
is $\mdscode{n}{\dualcodedim = n-t+1}{\F_p}$ code. Therefore (holds for all linear MDS  codes), every set of $n - t + 1$ rows of $H$ are independent.
Therefore, the resulting linear code $B_I$ \footnote{Jumping ahead, we will only consider $B_I$ for $I\in \{I_1,I_2\}$.}, projected onto any  subset $I\subseteq [n]$ with $|I|\geq n-t+1$, is the set of all vectors of the form $(\myvec{0},\vec{g},f^\bot(\vec{g}))$ where: 
\begin{itemize}
\item $\myvec{0}$ is a vector of 0's corresponding to the coordinate set $I\cap (I'_1\cup I'_2\cup I'_3)$.
\item $\vec{g}$ is an arbitrary vector in $\F^{|\freeind{I}{I'_1}{I'_2}{I'_3}|}_p$.
%\footnote{The fact that all vectors here appear follows from special properties of $C^\bot$ as dual to Shamir. It does not necessarily hold for any code with $H$ satisfying.} 

\item $f^\bot(\vec{g})$ is the (vector) value at coordinates $I\setminus {\left(\bigcup_{j \in [3]} I'_j\cup \freeind{I}{I'_1}{I'_2}{I'_3} \right)}$.
This value is obtained by applying the linear function $f^\bot$, determined by $C^\bot$, to $(\vec{0},\vec{g})$ (that together determine the codeword), complementing the required coordinates. 
\end{itemize}

Now, the set $H\cdot B_{I'_1,I'_2,I'_3}$ is a subset of $H\cdot B'_{I'_1,I'_2,I'_3}$, where additionally the codeword at coordinates $I_1\setminus I_i$ (similarly for the $I_2$ part) must be non-0. 
The transition from Equation~\ref{eq-8} to 
Equation~\ref{eq-9}, bounds each of the first two $\sqrt{\cdot}$ expressions by summing over $\beta$ in $B_{I'_1,I'_2,I'_3}$, by going over all $\vec{g}$ values not containing a $0$ coefficient, and \emph{assuming} that the resulting $f^\bot(\vec{g})$ also has no $0$ coefficients (taking the maximal possible coefficient as a bound for each).
However, for some such $\vec{g}$'s, additional $0$ coefficients may turn up in $f^\bot(\vec{g})$, so the contribution of that $\beta$ shouldn't have been accounted for (as $\beta\notin B_{I'_1,I'_2,I'_3}$). But, this may only increase the upper bound.
This only potentially increasing the expression, as all summands are non-negative (in particular, each $\vec{g}$ may appear at most once, as in $H\cdot B_{I'_1,I'_2,I'_3}$). A similar phenomena occurs in the part for responding to $I_2$, so the inequality follows.
Another fact we use in this transition, is that $\widehat{\myvec{1}}_{\ell_i}(0)=\frac{|A_{i,\ell_i}|}{p}$ for any boolean function $\myvec{1}_{\ell_i}$ we consider.

Further simplifying, we get:

\begin{align}
&S'_{I'_1,I'_2,I'_3} \nonumber\\ 
%%% A.13
&=\sum_{\ell} \prod_{i\in I'_1 \cup I'_2 \cup I'_3}\frac{|A_{i,\ell_i}|}{p} \cdot \prod_{\substack{i\in \freeione\cup\\ \; \freeitwo}} \sqrt{\frac{|A_{i,\ell_i}|}{p}\left(1-\frac{|A_{i,\ell_i}|}{p}\right)}\cdot \label{A.13}\\
&\hphantom{{}==\prod_{i\in I'_1 \cup I'_2 \cup I'_3}\left(\frac{|A_{i,\ell_i}|}{p}\right)}\prod_{\substack{i\in I_1\setminus{(\freeione\cup I'_1})\cup\\ \;\;\; I_2\setminus{(\freeitwo\cup I'_2})\cup\\  I_3\setminus{I'_3}}} \max_{\alpha\in\F_p\setminus{\{0\}}} |\widehat{\mathbf{1}}_{\ell_i}(\alpha)| \nonumber\\ \nonumber \\
%%% A.14 
&=\prod_{i\in I'_1 \cup I'_2 \cup I'_3} \; \sum_{\ell_i\in \{0,1\}}\frac{|A_{i,\ell_i}|}{p} \cdot \prod_{\substack{i\in\freeione\cup\\ \; \freeitwo}}     \left(\sum_{\ell_i\in\{0,1\}}\sqrt{\frac{|A_{i,\ell_i}|}{p}\left(1-\frac{|A_{i,\ell_i}|}{p}\right)}\right)\cdot \nonumber\\
&\hphantom{{}=\prod_{i\in I'_1 \cup I'_2 \cup I'_3} \; \sum_{\ell_i\in \{0,1\}}\frac{|A_{i,\ell_i}|}{p}}\prod_{\substack{ i\in I_1\setminus{(\freeione\cup I'_1})\cup\\ \;\;\; I_2\setminus{(\freeitwo\cup I'_1})\cup\\ I_3\setminus{I'_3}}} \left(\sum_{\ell_i\in\{0,1\}}\max_{\alpha\in\F_p\setminus{\{0\}}}|\widehat{\mathbf{1}}_{\ell_i}(\alpha)|\right) \nonumber\\ \nonumber \\
%%% A.15
&\leq c^{j + i + 2l + n - 2(n-t+1) - k}_1=O(c^{2t - n + i + j + l}_1) \label{A.15}
\end{align}

Here the equality~\ref{A.13} is simply using Parseval's identity, and the observation that $\widehat{\mathbf{1}}_{\ell_i}(0)=\frac{|A_{i,\ell_i}|}{p}$, and additionally simple arithmetic manipulation.
The inequality~\ref{A.15} relies on Claim~\ref{clm:tech} for upper bounding each multiplicand in the second product by 1.
Each multiplicand in the first product sums to 1, and each multiplicand in the third product is upper bounded by $c_1$, as follows from Lemma~\ref{lem:sumlibound}.
%Here $c_m = \frac{2^m\sin(\frac{\pi}{2^m})}{p\sin(\frac{\pi}{p})}$ is as defined in~\cite{EPRINT:BDIR19}.


Let us denote by
$$S'_{i,j,l}=\sum_{\substack{I'_1\subseteq I_1,|I_1|=i,\\
I'_2\subseteq I_2,|I_2|=j,\\I'_3\subseteq I_3,|I_3|=l}}S'_{I'_1,I'_2,I'_3} .$$ 
The number of summands $S'_{i,j,l}$ in the bound~\ref{eq:A.12}, rewritten in terms of the $S'_{i,j,l}$s, is only $poly(n)$ ($O(n^2)$, to be precise). 
Thus to bound $\SD{\vec{L}(C)}{\vec{L}(\F^n_p)}$ we may as well bound $t$ based on the maximum among the sums:

\begin{align}
  \log{\SD{\vec{L}(C)}{\vec{L}(\F^n_p)}} &\leq \log \left(\sum_{i+j+l\leq n-t+1}S'_{i,j,l} \right) \nonumber \\
  &\leq \log \left(O \left((n-t+1)^3 \right) \max_{i,j,l} S'_{i,j,l} \right) \nonumber \\
  &\leq \log{\max_{i,j,l}S'_{i,j,l}} + O(\log{n}) \nonumber
\end{align}

Unless stated otherwise, here and elsewhere $\log$ stands for $\log_2$.
As the bound on the maximum $\log(S'_{i,j,l})$ will be $\Omega(n)$ (for any $t = c\cdot n$ for constant $c$), we may indeed
search for $t$ for which $\log \left( \max_{i,j}S'_{i,j}\right) = -\Omega(n)$ (for arbitrarily small non-0 hidden constants, so the $O(\log(n))$ factor indeed has no effect).

Let us fix some $n-t+1=\gamma n$ for some constant $\gamma\in (0,1)$. We want to find the range of $\gamma$ values for which $\max\limits_{i+j+l\leq n-t+1}S'_{i,j,l}$ is $2^{-\Omega(n)}$.
Let $p$ and $n$ both go to infinity. The above requirement translates into the following,
\begin{align}
&\log \left(\max_{i+j+l\leq n-t+1}S'_{i,j,l} \right) \nonumber\\ \nonumber \\ 
&\leq \max_{i+j+l\leq n-t+1} \Bigg(\log \bigg(\sum_{\substack{
I'_1\subseteq I_1,|I_1|=i,\\
I'_2\subseteq I_2,|I_2|=j,\\
I'_3\subseteq I_3,|I_3|=l}}S'_{I'_1,I'_2,I'_3} \bigg) \Bigg) \nonumber\\ \nonumber\\
&\leq \max_{i+j+l\leq n-t+1} \Bigg(\log \left(\binom{n-t+1}{i} \cdot \binom{n-t+1}{j} \cdot \binom{2t - 2 - n}{l} \cdot \max_{I'_1,I'_2,I'_3}S'_{I'_1,I'_2,I'_3} \right) \Bigg)\label{eq-15}\\ \nonumber \\
&\leq n\cdot \Big(\gamma \cdot \Ent{a_1} + \gamma\cdot \Ent{a_2} + (1-2\gamma)\cdot \Ent{a_3\cdot b/(1-2\gamma)}- \nonumber \\
& \hphantom{{}=n\cdot \big(\gamma} \big(1 + \gamma\cdot(a_1 + a_2 + a_3 - 2)\big)\log(c_1) \Big)+\mathrm{poly}\log(n)
\end{align} 

Here $a_1,a_2,a_3$ denote
$\frac{i}{n-t+1},\frac{j}{n-t+1},\frac{l}{n-t+1}$ respectively. 
In the last step, we use Stirling's approximation, $\log(n!)=n\log(n) - n \log e+ O(\log(n))$, neglecting the $O(\log(n))$ term. Also, we replace $c_1$ with its limit $\lim_{p\rightarrow \infty}$. In this case, $c_1$ tends (from above) to $\log(2/\pi)$. %With the bound function above being continuous, we will conclude that any $\gamma$ smaller than a certain $\gamma_0$ . 
We conclude that the set of $\gamma$'s satisfying Equation~\ref{eq-l} below, result in $(n,(1-\gamma)n)$-Shamir being Leakage resilient, as required in Theorem~\ref{shamir}:% (the expression below is a bound on the log of the leakage error):
\begin{align}
&\max_{a_1,a_2,a_3} \Big(\Ent{a_1}+\Ent{a_2}+(1-2\gamma)\cdot \Ent{a_3\cdot \gamma/(1-2\gamma)}- \label{eq-l} \\
&\hphantom{{}=n\cdot \big(\Ent{a_1}+\Ent{}} \big(1 + \gamma\cdot(a_1 + a_2 + a_3 - 2)\big)\log(2/\pi) \Big) < 0 \nonumber
\end{align} 

In particular, the left hand side is well defined for all $0<\gamma<0.5$.
Fix some $\gamma$, using standard multi-variate analytic techniques (on the expressions as a function of $a_1,a_2,a_3$) over the domain $\{(a_1,a_2,a_3)|a_1+a_2+a_3\leq 1,a_1,a_2,a_3\geq 0\}$, we get that $\gamma\leq 0.133n$ leads to a negative value.

\noindent This concludes the proof of Theorem~\ref{shamir}.


%----------------------------------------------------------------------------------------
%	BIBLIOGRAPHY
%-----------------------------------------------------------------------
\printbibliography[heading=bibintoc]

%----------------------------------------------------------------------------------------

\end{document}  
