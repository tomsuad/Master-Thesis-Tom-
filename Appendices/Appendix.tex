% Appendix Template
\chapter{Appendix}

\section{Proof of Theorem \ref{shamir}} % Main appendix title
\label{proof-shamir}
%\label{AppendixX} % Change X to a consecutive letter; for referencing this appendix elsewhere, use \ref{AppendixX}

Let $C$ be a linear $\mdscode{n+1}{\codedim}{\F_p}$ code. The scheme induces a $(n,t=\codedim)$ linear secret sharing scheme $Sh_C$ as explained in Section~\ref{sec:prelim}.
As mentioned in Section~\ref{sec:prelim}, the leakage advantage of the scheme is upper bounded by $\SD{\vec{L}(C)}{\vec{L}(\F^n_p)}$, which we bound next. 

We start as in~\cite{EPRINT:BDIR19}, in the proof of their Lemma 4.18. These (identical)
calculations appear in the first three lines of the derivation below. Here $I_1,I_2$ are disjoint sets of coordinates of size $n-t+1$ each, and $I_3=[n]\setminus{(I_1\cup I_2)}$. For $\beta\in 
\F^{n-t+1}_p\setminus{\{0\}}$ we define $K_j=\{k' \in I_j | \ip{\beta,h_{k'}}=0 \}$.
%, and for $I'_j \subseteq I_j$ we define $K'_j=\{k \in I'_j | \ip{\beta,h_k}=0 \}$.

{\allowdisplaybreaks
\begin{align}
%%% A.1
&\SD{\vec{L}(C)}{\vec{L}(\F^n_p)}=\sum_{\vec{\ell}} \left|\sum_{\alpha\in C^{\bot}\setminus{\{0\}}}\prod^n_{i=1}\widehat{\mathbf{1}}_{\ell_i}(\alpha_i) \right|\label{eq:2}\\ &\nonumber \\
%%% A.2
&=\sum_{\vec{\ell}}\left|\sum_{\beta\in \F^{n-t+1}_p\setminus{\{0\}}}\prod^n_{i=1}\widehat{\mathbf{1}}_{\ell_i}(\ip{\beta,h_i})\right|\label{eq:3}\\ &\nonumber \\
%%% A.3  
&= \sum_{\vec{\ell}}\left|\sum_{\beta\in  \F^{n-t+1}_p\setminus{\{0\}}}\left(\prod_{i\in I_1}\widehat{\mathbf{1}}_{\ell_i}(\ip{\beta,h_i})\right)\cdot \left(\prod_{i\in I_2}\widehat{\mathbf{1}}_{\ell_i}(\ip{\beta,h_i})\right)\cdot \left(\prod_{i\in I_3}\widehat{\mathbf{1}}_{\ell_i}(\ip{\beta,h_i})\right)\right|\label{eq:4}\\ &\nonumber \\
%%%18
%&\leq \sum_{\vec{\ell}}\sum_{0\leq i+j+l\leq n-t+1} \\
%&\hphantom{{}=\leq \sum_{\vec{\ell}}}\sum_{\substack{\beta\in \F^{n-t+1}_p\setminus{\{0\}}\label{eq:5} s.t. \\ |K_1|=i,\\ |K_2|=j,\\ |K_3|=l}} \left|\left(\prod_{x\in I_1}\hat{\mathbf{1}_{\ell_x}}(\ip{\beta,h_x})\right)\cdot \left(\prod_{y\in I_2}\hat{\mathbf{1}_{\ell_y}}(\ip{\beta,h_y})\right)\cdot \left(\prod_{z\in I_3}\hat{\mathbf{1}_{\ell_z}}(\ip{\beta,h_z})\right) \right| \nonumber \\ &\nonumber \\
%%% A.4
&\leq \sum_{\vec{\ell}}\sum_{0\leq i+j+l\leq n-t+1} \; \sum_{\substack{\beta\in \F^{n-t+1}_p\setminus{\{0\}}\label{eq:5} \\ s.t. \; |K_1|=i,\\ \indent |K_2|=j,\\ \indent |K_3|=l}} \Bigg|\left(\prod_{x\in I_1}\widehat{\mathbf{1}}_{\ell_x}(\ip{\beta,h_x})\right)\cdot \\
& \hphantom{{}=\sum_{\vec{\ell}}\sum_{0\leq i+j+l\leq n-t+1} \sum_{\substack{\beta\in \F^{n-t+1}_p\setminus{\{0\}} s.t. \\ |K_1|=i,\\ |K_2|=j,\\ |K_3|=l}}} \left(\prod_{y\in I_2}\widehat{\mathbf{1}}_{\ell_y}(\ip{\beta,h_y})\right)\cdot \left(\prod_{z\in I_3}\widehat{\mathbf{1}}_{\ell_z}(\ip{\beta,h_z})\right) \Bigg| \nonumber \\ &\nonumber \\
%%%19
%&= \sum_{\vec{\ell}}\sum_{0\leq i+j+l\leq n-t+1} \sum_{\substack{I'_1\subseteq I_1,|I'_1|=i\\ \;\;\;\;I'_2\subseteq I_2,|I'_2|=j\\ \;\;\;\;I'_3\subseteq I_3,|I'_3|=l\\}}\\ &\hphantom{{}==\sum_{\vec{\ell}}\sum_{0\leq i+j+l\leq n-t+1}} \sum_{\substack{\beta\in \F^{n-t+1}_p\setminus{\{0\}}\label{eq:6} s.t.\\\forall k\in K'_1\\ \forall k\in K'_2\\ \forall k\in K'_3}} \left|\left(\prod_{x\in I_1}\hat{\mathbf{1}_{\ell_x}}(\ip{\beta,h_x})\right)\cdot \left(\prod_{y\in I_2}\hat{\mathbf{1}_{\ell_y}}(\ip{\beta,h_y})\right)\cdot \left(\prod_{z\in I_3}\hat{\mathbf{1}_{\ell_z}}(\ip{\beta,h_z})\right)\right|\nonumber \\ &\nonumber \\
%%% A.5
&= \sum_{\vec{\ell}}\sum_{0\leq i+j+l\leq n-t+1} \; \sum_{\substack{I'_1\subseteq I_1,|I'_1|=i\\ I'_2\subseteq I_2,|I'_2|=j\\ I'_3\subseteq I_3,|I'_3|=l\\}} \; \sum_{\substack{\beta\in \F^{n-t+1}_p\setminus{\{0\}}\label{eq:6}\\ s.t.\;K_1=I'_1\\ \indent K_2=I'_2\\ \indent K_3=I'_3}} 
\Bigg|
\left(\prod_{x\in I_1}\widehat{\mathbf{1}}_{\ell_x}(\ip{\beta,h_x})\right)\cdot \\
& \hphantom{{}=\sum_{\vec{\ell}}\sum_{0\leq i+j+l\leq n-t+1} \sum_{\substack{I'_1\subseteq I_1,|I'_1|=i\\ I'_2\subseteq I_2,|I'_2|=j\\ I'_3\subseteq I_3,|I'_3|=l\\}} \sum_{\substack{\beta\in \F^{n-t+1}_p\setminus{\{0\}} s.t.\\ K_1=I'_1\\ K_2=I'_2\\ K_3=I'_3}}} 
\left(\prod_{y\in I_2}\widehat{\mathbf{1}}_{\ell_y}(\ip{\beta,h_y})\right)\cdot 
\left(\prod_{z\in I_3}\widehat{\mathbf{1}}_{\ell_z}(\ip{\beta,h_z})\right)
\Bigg|\nonumber \\ 
&\nonumber \\
%%% A.6
&= \sum_{\vec{\ell}}\sum_{0\leq i+j+l\leq n-t+1} \; \sum_{\substack{I'_1\subseteq I_1,|I'_1|=i\\ I'_2\subseteq I_2,|I'_2|=j\\ I'_3\subseteq I_3,|I'_3|=l\\}}S_{I'_1,I'_2,I'_3}
%%%21
\end{align}}

Recall $H$ is a matrix who's columns span $C^\bot$, and let $h_j$ denote the $j$'s row of $H$.
Similarly to~\cite{EPRINT:BDIR19}, the next step is to bound the last expression in the chain using the Cauchy-Schwarz inequality.
The main innovation we introduce is splitting the sum in the expression into several sums as in Equation~\ref{eq:5}, and applying the bound separately to each sum.
Note the implicit definition of $S_{I'_1,I'_2,I'_3}$ in the last line.

\noindent Let us denote 
%\[B_{i,j,l}=\{\beta\in C^\bot\setminus{\{0\}}|\{k\in I_1|\ip{\beta,h_k}=0\}=i, |\{k\in I_2|\ip{\beta,h_k}=0\}|=j,|\{k\in I_3|\ip{\beta,h_k}=0\}|=l\}.\] 
$$B_{i,j,l}=\big\{\beta\in  \F^{n-t+1}_p\setminus{\{0\}} \Big| \left| K_1 \right|=i, \left| K_2 \right|=j, \left| K_3 \right|=l\big\}.$$  
\noindent We further denote 
% $B_{I'_1,I'_2,I'_3}=\{\beta\in B_{i,j,l} \Big| \forall k\in [3]\{u\in I_k|\ip{\beta,h_u}=0\}=I'_k\}$.
$$B_{I'_1,I'_2,I'_3}=\big\{\beta\in B_{i,j,l} \Big| I'_1=K_1, I'_2=K_2, I'_3=K_3 \big\}. $$

%We denote the subset $B'_{I'_1,I'_2,I'_3}$ of %$B_{I'_1,I'_2,I'_3}$ where $\bigcup_{i} I'_i$ is the \emph{exact} 

\noindent Rewriting the above inequality, we get

{\allowdisplaybreaks
\begin{align}
&\SD{\vec{L}(C)}{\vec{L}(\F^n_p)} \nonumber\\
%%% A.7
&\leq \sum_{\vec{\ell}} \sum_{0\leq i+j+l\leq n-t+1} \; \sum_{ \substack{\beta\in B_{i,j,l} \\ s.t.\; |K_1|=i,\\ \indent |K_2|=j,\\ \indent |K_3|=l}} \left| \left( \prod_{x\in I_1}\widehat{\mathbf{1}}_{\ell_x}(\ip{\beta,h_x}) \right) \cdot \left( \prod_{y\in I_2}\widehat{\mathbf{1}}_{\ell_y}(\ip{\beta,h_y}) \right)	\right| \cdot \label{eq-5} \\
&\hphantom{{}=\leq \sum_{\vec{\ell}} \sum_{0\leq i+j+l\leq n-t+1} \sum_{	\substack{\beta\in B_{i,j,l} s.t.\\ |K_1|=i,\\ |K_2|=j,\\|K_3|=l}}}\max_{\beta\in B_{i,j,l}} \left| \prod_{z\in I_3} \widehat{\mathbf{1}}_{\ell_z}(\ip{\beta,h_z}) \right| \nonumber \\ &\nonumber \\
%%% A.8
&\leq \sum_{\vec{\ell}} \sum_{0\leq i+j+l\leq n-t+1} \; \sum_{\substack{I'_1\subseteq I_1,|I'_1|=i, \\I'_2\subseteq I_2,|I'_2|=j, \\I'_3\subseteq I_3,|I'_3|=l}} \sqrt{\sum_{\substack{\beta\in B_{I'_1,I'_2,I'_3}}} \prod_{x\in I_1}|\widehat{\mathbf{1}}_{\ell_x}(\ip{\beta,h_x})|^2}\;\;\cdot \label{eq:7} \\
&\hphantom{{}=\leq \sum_{\vec{\ell}} \sum_{0\leq i+j+l\leq n-t+1} \sum_{\substack{I'_1\subseteq I_1,|I'_1|=i,\\I'_2\subseteq I_2,|I'_2|=j,\\I'_3\subseteq I_3,|I'_3|=l}}} \sqrt{\sum_{\substack{\beta\in B_{I'_1,I'_2,I'_3}}} \prod_{y\in I_2}|\widehat{\mathbf{1}}_{\ell_y}(\ip{\beta,h_y})|^2} \cdot\max_{\beta\in B_{i,j,l}}\left|\prod_{z\in I_3}\widehat{\mathbf{1}}_{\ell_z}(\ip{\beta,h_z})\right| \nonumber
%%%23
\end{align}}

The inequality~\ref{eq-5} is simply due to splitting the sum according to the \emph{exact} locations of $0$ coordinates in $H\beta^T$.
The inequality~\ref{eq:7} is by Cauchy-Schwartz, where the maximum is taken over a larger set than needed, which clearly only increases the bound.

To continue, we will use the following technical observation.

\begin{observation}
\label{clm:tech}
For sets $A_1,A_2$ with $|A_1|+|A_2|=p$, we have $\sum\limits_{i\in[2]}\sqrt{\sum\limits_{\alpha\neq 0}|\widehat{\vec{1}}_{A_i}(\alpha)|^2}\leq 1$.
\end{observation}

\begin{proof}
By Parseval's identity, we have
$||\widehat{\vec{1}}_{A_i}||^2_2 = ||\vec{1}_{A_i}||^2_2=\frac{|A_i|}{p}$. 
Let us denote $q_i=\frac{|A_i|}{p}$.
Thus
%\[||1_{A_i}||_2=\sqrt{\sum_{\alpha\neq 0}|\hat{A_i}(\alpha)|^2} = \sqrt{q_i - q^2_i},\]%
$$\sum_{\alpha\neq 0}|\widehat{\vec{1}}_{A_i}(\alpha)|^2 = q_i - q^2_i$$
as $|\widehat{\vec{1}}_{A_i}(0)|=q_i$. Thus, we have
\begin{equation}
%%% A.9
\label{eq:1}
\sum^2_{i=1}\sqrt{\sum_{\alpha\neq 0}|\widehat{\vec{1}}_{A_i}(\alpha)|^2} = \sqrt{q_1 - q^2_1} + \sqrt{q_2 - q^2_2}
\end{equation}
We denote $g(x)=\sqrt{x(1-x)}$ and observe that the expression in Equation~\ref{eq:1} equals $g(q_1)+g(1-q_1)=2g(q_1)$ (as $q_2=(1-q_1)$ and $g(q) = g(1-q)$).
The maximum of $2g(q_1)$ in $[0,1]$ is easily seen to be obtained at $q_1=0.5$, and equal $1$.
\footnote{Part of our gain comes from a more careful analysis for the case of two sets. Here we compute the maximum exactly instead of using concavity arguments to bound it, as is done in~\cite{EPRINT:BDIR19}. This is not surprising, because they did not attempt to optimize bounds for the case of $m=1$.}
\end{proof}
Let us bound the contribution of the portion a single $I'_1,I'_2,I'_3$ contributed to the expression
in Equation~\ref{eq:7} - 
we denote this contribution by $S'_{I'_1,I'_2,I'_3}$. 

{\allowdisplaybreaks
\begin{align}
&S'_{I'_1,I'_2,I'_3}\nonumber\\ \nonumber \\
%%% A.10
&=\sum_{\vec{\ell}} \sqrt{\sum_{\substack{\beta\in B_{I'_1,I'_2,I'_3}}} \prod_{x\in I_1}|\widehat{\mathbf{1}}_{\ell_x}(\ip{\beta,h_x})|^2}\cdot \sqrt{\sum_{\substack{\beta\in B_{I'_1,I'_2,I'_3}}} \prod_{y\in I_2}|\widehat{\mathbf{1}}_{\ell_y}(\ip{\beta,h_y})|^2} \cdot \label{eq-8} \\ \nonumber
&\hphantom{{}=\sum_{\vec{\ell}}} \max_{\beta\in B_{i,j,l}}\left| \prod_{z\in I_3}\widehat{\mathbf{1}}_{\ell_z}(\ip{\beta,h_z})\right| \\ \nonumber \\
%%% A.11
&\leq \sum_{\vec{\ell}} \sqrt{ \prod_{x\in I'_1}(\frac{|A_x|}{p})^2 \cdot \prod_{x\in \freeione}\left(\sum_{\alpha\in\F_p\setminus{\{0\}}} |\widehat{\mathbf{1}}_{\ell_x}(\alpha)|^2\right)\cdot \prod_{x\in I_1\setminus{(\freeione\cup I'_1})} \max_{\alpha\in\F_p\setminus{\{0\}}}|\widehat{\mathbf{1}}_{\ell_x}(\alpha)|^2 \label{eq-9}}\cdot\\\nonumber
&\hphantom{{}=\sum_{\vec{\ell}}}\sqrt{ \prod_{y\in I'_2}(\frac{|A_y|}{p})^2 \cdot \prod_{y\in \freeitwo}\left(\sum_{\alpha\in\F_p\setminus{\{0\}}} |\widehat{\mathbf{1}}_{\ell_y}(\alpha)|^2\right)\cdot \prod_{y\in I_2\setminus{(\freeitwo\cup I'_2})} \max_{\alpha\in\F_p\setminus{\{0\}}}|\widehat{\mathbf{1}}_{\ell_y}(\alpha)|^2\label{eq-10} }\cdot\\\nonumber
&\hphantom{{}=\sum_{\vec{\ell}}}\prod_{z\in I'_3}(\frac{|A_z|}{p}) \cdot \prod_{z\in I_3\setminus{I'_3}}|\max_{\alpha\in \F_p\setminus{\{0\}}}\widehat{\mathbf{1}}_{\ell_z}(\alpha)|
\end{align}}

\noindent Therefore we have
\begin{align} \label{eq:A.12}
%%% A.12
\SD{\vec{L}(C)}{\vec{L}(\F^n_p)} \leq \sum_{\substack{(I'_1,I'_2,I'_3)\\ s.t. \; \forall j \in [3] \; I'_j\subseteq I_j, \\ \;\;\;\; | \bigcup_{j \in [3]} I'_j | \leq n-t+1}} S'_{I'_1,I'_2,I'_3}
%|\bigcup\limits_{j \in [3]} I'_j|\leq n-t+1}S'_{I'_1,I'_2,I'_3}
%|I'_1 \cup I'_2 \cup I'_3|
\end{align}

We start by separately bounding each of the sums $S'_{I'_1,I'_2,I'_3}$.

\noindent We define the notation $\freeind{\cdot}{\cdot}{\cdot}{\cdot}$ and explain the above sequence of inequalities next. 

For $I\subseteq [n]$ satisfying $|I|\geq n-t+1$ and $I'_1,I'_2,I'_3$
with $|\cup_{j\in [3]}I'_j|\leq n-t+1$, we let $\freeind{I}{I'_1}{I'_2}{I'_3}$ denote a subset of $I\setminus{\cup_{j\in [3]}I'_j}$ of size $n-t+1-|\cup_{j\in [3]}I'_j|$ (picked arbitrarily, say according to lexicographic order). When $I'_1,I'_2,I'_3$ are clear from the context, we abbreviate $\freei=\freeind{I_i}{I'_1}{I'_2}{I'_3}$.

Consider a superset $B'_{I'_1,I'_2,I'_3}$ of $B_{I'_1,I'_2,I'_3}$ consisting of all $\beta\in \F^{n-t+1}_p$ such that
$H\beta^T$ has 0's at all coordinates $\bigcup_{j \in [3]} I'_j$, but may as well have $0$'s \emph{elsewhere}.
Restricting $\beta$ to $B'_{I'_1,I'_2,I'_3}$
results in a linear code $B'=\{H\beta^T|\beta\in B'_{I'_1,I'_2,I'_3}\}$.
From now on, for a set $B\subseteq \F^{n-t+1}_p$ we abbreviate the multi-set 
$\{H\beta^T|\beta\in B\}$ by $H\cdot B$.
It is well known that $C$ corresponding to $(n,t)$-Shamir secret sharing scheme (which is a special case of a Massey code) is $\mdscode{n}{\codedim=t-1}{\F_p}$ code. Therefore, $C^\bot$ 
is $\mdscode{n}{\dualcodedim = n-t+1}{\F_p}$ code. Therefore (holds for all linear MDS  codes), every set of $n - t + 1$ rows of $H$ are independent.
Therefore, the resulting linear code $B_I$ \footnote{Jumping ahead, we will only consider $B_I$ for $I\in \{I_1,I_2\}$.}, projected onto any  subset $I\subseteq [n]$ with $|I|\geq n-t+1$, is the set of all vectors of the form $(\myvec{0},\vec{g},f^\bot(\vec{g}))$ where: 
\begin{itemize}
\item $\myvec{0}$ is a vector of 0's corresponding to the coordinate set $I\cap (I'_1\cup I'_2\cup I'_3)$.
\item $\vec{g}$ is an arbitrary vector in $\F^{|\freeind{I}{I'_1}{I'_2}{I'_3}|}_p$.
%\footnote{The fact that all vectors here appear follows from special properties of $C^\bot$ as dual to Shamir. It does not necessarily hold for any code with $H$ satisfying.} 

\item $f^\bot(\vec{g})$ is the (vector) value at coordinates $I\setminus {\left(\bigcup_{j \in [3]} I'_j\cup \freeind{I}{I'_1}{I'_2}{I'_3} \right)}$.
This value is obtained by applying the linear function $f^\bot$, determined by $C^\bot$, to $(\vec{0},\vec{g})$ (that together determine the codeword), complementing the required coordinates. 
\end{itemize}

Now, the set $H\cdot B_{I'_1,I'_2,I'_3}$ is a subset of $H\cdot B'_{I'_1,I'_2,I'_3}$, where additionally the codeword at coordinates $I_1\setminus I_i$ (similarly for the $I_2$ part) must be non-0. 
The transition from Equation~\ref{eq-8} to 
Equation~\ref{eq-9}, bounds each of the first two $\sqrt{\cdot}$ expressions by summing over $\beta$ in $B_{I'_1,I'_2,I'_3}$, by going over all $\vec{g}$ values not containing a $0$ coefficient, and \emph{assuming} that the resulting $f^\bot(\vec{g})$ also has no $0$ coefficients (taking the maximal possible coefficient as a bound for each).
However, for some such $\vec{g}$'s, additional $0$ coefficients may turn up in $f^\bot(\vec{g})$, so the contribution of that $\beta$ shouldn't have been accounted for (as $\beta\notin B_{I'_1,I'_2,I'_3}$). But, this may only increase the upper bound.
This only potentially increasing the expression, as all summands are non-negative (in particular, each $\vec{g}$ may appear at most once, as in $H\cdot B_{I'_1,I'_2,I'_3}$). A similar phenomena occurs in the part for responding to $I_2$, so the inequality follows.
Another fact we use in this transition, is that $\widehat{\myvec{1}}_{\ell_i}(0)=\frac{|A_{i,\ell_i}|}{p}$ for any boolean function $\myvec{1}_{\ell_i}$ we consider.

Further simplifying, we get:

\begin{align}
&S'_{I'_1,I'_2,I'_3} \nonumber\\ 
%%% A.13
&=\sum_{\ell} \prod_{i\in I'_1 \cup I'_2 \cup I'_3}\frac{|A_{i,\ell_i}|}{p} \cdot \prod_{\substack{i\in \freeione\cup\\ \; \freeitwo}} \sqrt{\frac{|A_{i,\ell_i}|}{p}\left(1-\frac{|A_{i,\ell_i}|}{p}\right)}\cdot \label{A.13}\\
&\hphantom{{}==\prod_{i\in I'_1 \cup I'_2 \cup I'_3}\left(\frac{|A_{i,\ell_i}|}{p}\right)}\prod_{\substack{i\in I_1\setminus{(\freeione\cup I'_1})\cup\\ \;\;\; I_2\setminus{(\freeitwo\cup I'_2})\cup\\  I_3\setminus{I'_3}}} \max_{\alpha\in\F_p\setminus{\{0\}}} |\widehat{\mathbf{1}}_{\ell_i}(\alpha)| \nonumber\\ \nonumber \\
%%% A.14 
&=\prod_{i\in I'_1 \cup I'_2 \cup I'_3} \; \sum_{\ell_i\in \{0,1\}}\frac{|A_{i,\ell_i}|}{p} \cdot \prod_{\substack{i\in\freeione\cup\\ \; \freeitwo}}     \left(\sum_{\ell_i\in\{0,1\}}\sqrt{\frac{|A_{i,\ell_i}|}{p}\left(1-\frac{|A_{i,\ell_i}|}{p}\right)}\right)\cdot \nonumber\\
&\hphantom{{}=\prod_{i\in I'_1 \cup I'_2 \cup I'_3} \; \sum_{\ell_i\in \{0,1\}}\frac{|A_{i,\ell_i}|}{p}}\prod_{\substack{ i\in I_1\setminus{(\freeione\cup I'_1})\cup\\ \;\;\; I_2\setminus{(\freeitwo\cup I'_1})\cup\\ I_3\setminus{I'_3}}} \left(\sum_{\ell_i\in\{0,1\}}\max_{\alpha\in\F_p\setminus{\{0\}}}|\widehat{\mathbf{1}}_{\ell_i}(\alpha)|\right) \nonumber\\ \nonumber \\
%%% A.15
&\leq c^{j + i + 2l + n - 2(n-t+1) - k}_1=O(c^{2t - n + i + j + l}_1) \label{A.15}
\end{align}

Here the equality~\ref{A.13} is simply using Parseval's identity, and the observation that $\widehat{\mathbf{1}}_{\ell_i}(0)=\frac{|A_{i,\ell_i}|}{p}$, and additionally simple arithmetic manipulation.
The inequality~\ref{A.15} relies on Claim~\ref{clm:tech} for upper bounding each multiplicand in the second product by 1.
Each multiplicand in the first product sums to 1, and each multiplicand in the third product is upper bounded by $c_1$, as follows from Lemma~\ref{lem:sumlibound}.
%Here $c_m = \frac{2^m\sin(\frac{\pi}{2^m})}{p\sin(\frac{\pi}{p})}$ is as defined in~\cite{EPRINT:BDIR19}.


Let us denote by
$$S'_{i,j,l}=\sum_{\substack{I'_1\subseteq I_1,|I_1|=i,\\
I'_2\subseteq I_2,|I_2|=j,\\I'_3\subseteq I_3,|I_3|=l}}S'_{I'_1,I'_2,I'_3} .$$ 
The number of summands $S'_{i,j,l}$ in the bound~\ref{eq:A.12}, rewritten in terms of the $S'_{i,j,l}$s, is only $poly(n)$ ($O(n^2)$, to be precise). 
Thus to bound $\SD{\vec{L}(C)}{\vec{L}(\F^n_p)}$ we may as well bound $t$ based on the maximum among the sums:

\begin{align}
  \log{\SD{\vec{L}(C)}{\vec{L}(\F^n_p)}} &\leq \log \left(\sum_{i+j+l\leq n-t+1}S'_{i,j,l} \right) \nonumber \\
  &\leq \log \left(O \left((n-t+1)^3 \right) \max_{i,j,l} S'_{i,j,l} \right) \nonumber \\
  &\leq \log{\max_{i,j,l}S'_{i,j,l}} + O(\log{n}) \nonumber
\end{align}

Unless stated otherwise, here and elsewhere $\log$ stands for $\log_2$.
As the bound on the maximum $\log(S'_{i,j,l})$ will be $\Omega(n)$ (for any $t = c\cdot n$ for constant $c$), we may indeed
search for $t$ for which $\log \left( \max_{i,j}S'_{i,j}\right) = -\Omega(n)$ (for arbitrarily small non-0 hidden constants, so the $O(\log(n))$ factor indeed has no effect).

Let us fix some $n-t+1=\gamma n$ for some constant $\gamma\in (0,1)$. We want to find the range of $\gamma$ values for which $\max\limits_{i+j+l\leq n-t+1}S'_{i,j,l}$ is $2^{-\Omega(n)}$.
Let $p$ and $n$ both go to infinity. The above requirement translates into the following,
\begin{align}
&\log \left(\max_{i+j+l\leq n-t+1}S'_{i,j,l} \right) \nonumber\\ \nonumber \\ 
&\leq \max_{i+j+l\leq n-t+1} \Bigg(\log \bigg(\sum_{\substack{
I'_1\subseteq I_1,|I_1|=i,\\
I'_2\subseteq I_2,|I_2|=j,\\
I'_3\subseteq I_3,|I_3|=l}}S'_{I'_1,I'_2,I'_3} \bigg) \Bigg) \nonumber\\ \nonumber\\
&\leq \max_{i+j+l\leq n-t+1} \Bigg(\log \left(\binom{n-t+1}{i} \cdot \binom{n-t+1}{j} \cdot \binom{2t - 2 - n}{l} \cdot \max_{I'_1,I'_2,I'_3}S'_{I'_1,I'_2,I'_3} \right) \Bigg)\label{eq-15}\\ \nonumber \\
&\leq n\cdot \Big(\gamma \cdot \Ent{a_1} + \gamma\cdot \Ent{a_2} + (1-2\gamma)\cdot \Ent{a_3\cdot b/(1-2\gamma)}- \nonumber \\
& \hphantom{{}=n\cdot \big(\gamma} \big(1 + \gamma\cdot(a_1 + a_2 + a_3 - 2)\big)\log(c_1) \Big)+\mathrm{poly}\log(n)
\end{align} 

Here $a_1,a_2,a_3$ denote
$\frac{i}{n-t+1},\frac{j}{n-t+1},\frac{l}{n-t+1}$ respectively. 
In the last step, we use Stirling's approximation, $\log(n!)=n\log(n) - n \log e+ O(\log(n))$, neglecting the $O(\log(n))$ term. Also, we replace $c_1$ with its limit $\lim_{p\rightarrow \infty}$. In this case, $c_1$ tends (from above) to $\log(2/\pi)$. %With the bound function above being continuous, we will conclude that any $\gamma$ smaller than a certain $\gamma_0$ . 
We conclude that the set of $\gamma$'s satisfying Equation~\ref{eq-l} below, result in $(n,(1-\gamma)n)$-Shamir being Leakage resilient, as required in Theorem~\ref{shamir}:% (the expression below is a bound on the log of the leakage error):
\begin{align}
&\max_{a_1,a_2,a_3} \Big(\Ent{a_1}+\Ent{a_2}+(1-2\gamma)\cdot \Ent{a_3\cdot \gamma/(1-2\gamma)}- \label{eq-l} \\
&\hphantom{{}=n\cdot \big(\Ent{a_1}+\Ent{}} \big(1 + \gamma\cdot(a_1 + a_2 + a_3 - 2)\big)\log(2/\pi) \Big) < 0 \nonumber
\end{align} 

In particular, the left hand side is well defined for all $0<\gamma<0.5$.
Fix some $\gamma$, using standard multi-variate analytic techniques (on the expressions as a function of $a_1,a_2,a_3$) over the domain $\{(a_1,a_2,a_3)|a_1+a_2+a_3\leq 1,a_1,a_2,a_3\geq 0\}$, we get that $\gamma\leq 0.133n$ leads to a negative value.

\noindent This concludes the proof of Theorem~\ref{shamir}.